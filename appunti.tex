\documentclass[letterpaper,10pt]{article}

\usepackage{sectsty}
\usepackage{graphicx}
% Margins
% \topmargin=-0.45in
% \evensidemargin=0in
% \oddsidemargin=0in
% \textwidth=6.5in
% \textheight=9.0in
% \headsep=0.25in

\title{Economia dei Sistemi Industriali }
\author{ Emanuele Micheletti }
\date{2022}

\begin{document}

\maketitle
\tableofcontents

%--Introduzione--

\section{Introduzione}
\subsection{Contatti}

Carlo Cambini: \url{carlo.cambini@polito.it}
Luigi Buzzacchi: luigi.buzzacchi@polito.it

\subsection{Bibliografia}
This subsection's content...

\subsection{Regole d'esame}

\subsection{Descrizione del corso}
Il corso Economia dei Sistemi Industriali tratta principalmente l'organizzazione dei mercati e le interazioni tra i vari operatori, sia dalla parte della domanda che dalla parte dell'offerta in presenza di diverse rotture di mercato e al variare di determinate variabili come il tempo. Questa tipologia di mercato è ben descritto dall'oligopolio: si tratta infatti di un approccio più sofisticato rispetto alla descrizione effettuata dalle teorie dei mercati perfetti microeconomici, verrá quindi trattato soltanto l'intermezzo tra la concorrenza perfetta e il monopolio. In questa situazione le imprese hanno la capacitá di influenzare il mercato ma non di controllarlo e al tempo stesso le loro scelte sono condizionate dalla concorrenza che risulta quindi "non perfetta".

Dopo un piccolo ripasso introduttivo verrá affrontata la teoria dei giochi che spiega come le performance di un'impresa dipendono da moltissimi fattori in gioco che performano tutti nello stesso momento. Successivamente ci si occuperá della regolamentazione ovvero degli interventi istituzionali che permettono, o che dovrebbero permettere, un corretto funzionamento del mercato imperfetto illustrando anche casi reali.



\pagebreak
\chapter{First}
This chapter's content...

%--/Paper--

\end{document}