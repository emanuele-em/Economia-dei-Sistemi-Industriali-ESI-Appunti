\chapter{Concorrenza perfetta, monopolio, potere e fallimenti di mercato}
\label{sec:Concorrenza perfetta, monopolio, potere e fallimenti di mercato}

\section{Concorrenza perfetta}
\begin{itemize}
    \item Aziende Price Taker: le imprese non possono fissare il prezzo, vendono al prezzo di mercato
    \item Il prezzo deriva dall'interazione tra domanda e offerta, l'unico in grado di fissarlo è quindi il mercato
    \item I prodotti sono omogenei
    \item L'informazione è perfetta
\end{itemize}

\begin{equation} \label{eq:1}
    p=(L)ML=(LR)AC
\end{equation}

Come vediamo nell'equazione \ref{eq:1} i prezzi vengono fissati ai costi marginali incrementali, nel lungo periodo si dimostra che il prezzo è anche pari al costo medio e di conseguenza nel lungo periodo le imprese non hanno extra-profitti.

Durante il corso si considererá solamente la forma di mercato "concorrenza perfetta" perché è l'unica che massimizza il benessere collettivo (welfare). Favorire la concorrenza è quindi importante per avvicinarsi allo stato di benessere collettivo ideale.

Nell'immagine \ref{fig:curva_domanda_offerta} è possibile osservare la curva di domanda (decrescente) e la curva dell'offerta (crescente). Aumentando la domanda, spostando quindi la curva decrescente a destra,  li prezzo di equilibrio definito da \(p*\) si sposterebbe lungo la curva dell'offerta verso destra aumentando. Viceversa aumentando l'offerta il prezzo \(p*\) si sposterebbe lungo la curva della domanda diminuendo.
\begin{figure}[h]
    \centering
    \includegraphics[width=.7\linewidth]{images/chapter1/1.jpg}
    \caption{Curva di domanda e offerta}
    \label{fig:curva_domanda_offerta}
\end{figure}

\section{Domanda individuale: teoria del consumatore}

\begin{equation}
    \label{eq:2}
    \begin{gathered}
        \nu(p,m)=max(u(x)) \\
        s.t. \underbrace{p*q = m}_{\makebox[0pt]{Vincolo di bilancio}}
    \end{gathered}
\end{equation}
\begin{conditions*}
    p*q &   spesa del consumatore \\
    m   &   reddito
\end{conditions*}

Il vincolo di bilancio quindi tiene conto di reddito e prezzi, risolvendo l'equazione trovo \(q\) cioè la quantità. Trovando il massimo della funzione di utilità rispetto alla quantità trovo \(q'\).

Il problema di massimizzazione dell'utilità può essere risolto utilizzando la Lagrangiana:
\begin{equation}
    \label{eq:lagrange}
    L=u(q)-\lambda(p*q-m)
\end{equation}
\begin{conditions*}
    u(q)    &   Spesa del consumatore \\
    \lambda &   Reddito \\
    p*q-m   &   Vincolo di bilancio
\end{conditions*}

A questo punto si calcola la derivata dell'utilità e la si eguaglia a 0 per massimizzarla:
\begin{equation}
    \frac{\partial u(q)}{\partial q_i} - \lambda(p_i) = 0
\end{equation}

Che può essere riscritta come segue
\begin{equation}
    \underbrace{\frac{\frac{\partial u(q*)}{\partial q_i}}{\frac{\partial u(q*)}{\partial q_j}}}_{(a)} = \underbrace{\frac{p_i}{p_j}}_{(b)}
\end{equation}
\begin{conditions*}
    (a) &   saggio marginale di sostituzione, corrisponde all'inclinazione della curva di indifferenza \\
    (b) &   rapporto tra i prezzi che corrisponde all'inclinazione della curva di bilancio
\end{conditions*}

Questa condizione nasconde il rapporto tra effetto reddito ed effetto sostituzione, un questo corso si presuppone la condizione di \textbf{equilibrio parziale}.

\subsection{Funzione di utilità quasi-lineare}

Si considera un mercato preso singolarmente, idealmente isolato rispetto agli altri, in questo contesto \textbf{equilibrio parziale} significa che una variazione all'interno di un mercato impatta principalmente sul mercato stesso e non nell'economia globale.

Si introducono quindi 2 ipotesi semplificative:
\begin{enumerate}
    \item \label{itm:1}Effetto sostituzione nullo
    \item \label{itm:2}Effetto reddito nullo
\end{enumerate}

Il prezzo degli altri beni può essere considerato fisso e può essere normalizzato a "1". È possibile quindi , a fronte delle condizioni \ref{itm:1} and \ref{itm:2} riscrivere la funzione di utilità come segue:

\begin{equation}
    \label{eq:3}
    \begin{gathered}
        U(x,y)=u(x)+y \\
        s.t. p*x + y = m
    \end{gathered}
\end{equation}
\begin{conditions*}
    p*x &   spesa del consumatore \\
    y   &   quantità del bene y \\
    m   &   reddito
\end{conditions*}

Dopo aver calcolato la lagrangiana, come nell'equazione \ref{eq:lagrange}, massimizzo nuovamente l'utilità, sia in termini di x che in termini di y:

\begin{equation}
    \label{eq:4}
    \begin{split}
        &\frac{\partial L}{\partial x} = \frac{\partial u(x)}{\partial x} - \lambda*p = 0 \\
        &\frac{\partial L}{\partial y} = 1- \lambda = 0
    \end{split}
\end{equation}

Unendo le equazioni \ref{eq:4} ottengo:

\begin{equation}\label{eq:maxutilita}
    \frac{\partial u(x)}{\partial x} = u'(x) = p
\end{equation}
\begin{conditions*}
    u'(x)   &   utilità marginale \\
    p       &   prezzo
\end{conditions*}

Ne consegue che l'utilità marginale è uguale al prezzo.

In queste condizioni quindi la quantità \(x\) dipende solo e soltanto dal prezzo, non si ha pertanto effetto reddito.

L'interpretazione è quindi molto più diretta, il consumatore comprerà il bene fino a quando la sua utilità marginale sarà maggiore o uguale al prezzo.

\subsection{Surplus}

\begin{itemize}
    \item Surplus del consumatore: beneficio totale o valore che il consumatore riceve oltre al prezzo pagato.
    \item Surplus del produttore: beneficio totale o ricavo/profitto ch eun produttore riceve oltre ai costi di produzione
\end{itemize}

\begin{figure}[H]
    \centering
    \includegraphics[width=.6\linewidth]{images/chapter1/2.jpg}
    \caption{Surplus del consumatore (in giallo) e surplus del produttore (in blu)}
    \label{fig:curva_surplus}
\end{figure}

Tutte le aziende vendono un prodotto ad un prezzo uguale pari a 5.

\begin{itemize}
    \item I consumatore nel punto \(Q_d\): massimizzerebbero la loro utilità anche pagando 9, sarebbero quindi disposti a pagare 9 ma pagano 5, hanno quindi un surplus (del consumatore) pari a 4.
    \item I produttore in \(Q_s\) massimizzerebbero la loro utilià anche se vendessero ad un prezzo pari a 3, ricavano però 5 quindi hanno un surplus di 2.
\end{itemize}

È molto iportante capire come molte delle decisioni dipendono da questo surplus, sia dal lato del consumatore che dal lato del produttore.

Possiamo calcolare l'area dei due triangoli in due modi differenti:

\begin{enumerate}
    \item Dal lato del prezzo (guardando il grafico orizzontalmente)

          \begin{equation}
              CS = V(p) = \int_{p*}^{\infty}q(p)\,dp
          \end{equation}

          risulta quindi:
          \begin{equation}
              \frac{\partial V(p)}{\partial p} = -q(p)
          \end{equation}

          La domanda presenta quindi un'inclinazione negativa, il segno "-" è dovuto al fatto che \(p*\) è l'estremo inferiore. Se aumenta il prezzo, il surplus del consumatore diminuisce.

    \item Dal lato della quantitá (guardando il grafico verticalmente):
          \begin{equation}
              CS = S(q) - p(q)*q
          \end{equation}

          dove:
          \begin{equation}
              S(q) = \int_{0}^{q*}  p(q)\,dq
          \end{equation}
          \(S(q)\) è quindi il surplus lordo, ovvero il trapezio rettangolo che comprende il triangolo giallo, il triangolo blu e la base maggiore fino a \(Q_0\).

          Risulta quindi:
          \begin{equation}
              \frac{\partial S(q)}{\partial q} = p(q)
          \end{equation}

          \begin{conditions*}
              p(q)            &   spesa del consumatore\\
              \partial S(q)   &   Variazione del surplus lordo
          \end{conditions*}

          In questo caso se aumento la quantità l'area del surplus lordo aumenta.
\end{enumerate}

Ne consegue che il surplus del consumatore ha due definizioni differenti a seconda della variabile di riferimento utilizzata (prezzo o quantità).

\section{Economia del Benessere (Welfare)}
\subsection{Applicazioni in concorrenza perfetta}
Che effetto ha la presenza di un mercato in concorrenza perfetta sull'utilitá del consumatore?
\begin{itemize}
    \item 1 impresa in concorrenza perfetta
    \item 1 consuamatore
\end{itemize}
l'utilitá massimizzata è l'equazione \ref{eq:maxutilita} quindi:
\[u'(x)=p\]
L'impresa in questione ha un costo \(c'>0\), \(c''>0\) e \(c(0)=0\), in concorrenza perfetta quindi
\[p=c'(x)\]
Ne consegue che:
\begin{equation}
    u'(x)=c'(x)
\end{equation}
Questa condizione mi diche che il consumatore continuerà a consumare il bene fino a quando il beneficio che ottiene coprirá i costi di produzione di quel bene.

\subsection{Applicazione con funzione di Welfare}

Definiamo quindi la funzione di Welfare:
\begin{equation}
    \label{eq:welfare}
    \begin{split}
        W   &= max[CS(q) + PS(q)] =\\
            &= [u(q) - p(q)] + p*q - c*q =\\
            &= u(q) - c(q)
    \end{split}
\end{equation}
\begin{conditions*}
    q & quantitá
\end{conditions*}
Con lo sviluppo \ref{eq:welfare} si dimostra quindi che la concorrenza perfetta massimizza l'utilitá del consumatore e il benessere collettivo. È quindi considerata il caso \textbf{benchmark}.

\subsection{Primo teorema dell'economia del benessere}

\begin{theorem}
    Se tutti i mercati fossero in concorrenza perfetta ogni scambio porterebbe alla migliore allocazione delle risorse possibile e sarebbero, di conseguenza, economicamente efficienti.
\end{theorem}

\begin{theorem}
    Ogni allontanamento dalle condizioni di concorrenza perfetta ha, come conseguenza, un effetto peggiorativo sul benessere collettivo.\footnote{Un peggioramento del benessere collettivo può significare anche uno sbilanciamento come ad esempio un aumento del benessere delle sole imprese a scapito di una diminuzione del benessere dei consumatori o viceveversa}
\end{theorem}

\subsection{Equitá ed Efficienza}
L'equilibrio dei mercati porta anche ad un allocazione delle risorse equa a livello sociale?

La risposta è \textbf{no}, non c'è nessuna ragione per dire che un mercato efficiente porta ad un'allocazione delle risorse socialmente equa, da qui deriva il secondo teorema del benessere;

\subsection{Secondo teorema dell'economia del benessere}

\begin{theorem}
    La soluzione dell'economia perfetta porterebbe ad una soluzione equa dal punto di vista sociale solo sotto la condizione stringente che cci sia una riallocazione iniziale dei beni
\end{theorem}

Es. un ricco, prima di usufruire di un bene, dovrebbe distribuire i suoi averi alla collettività, in modo che poi, secondo le regole della concorrenza perfetta, tutti potrebbero usufruire in modo uguale di quello stesso bene.

Una soluzione parziale e incompleta è quella della tassazione.

\section{Fallimenti di mercato}

Rappresentano tutte le cause che rendono il mercato imperfetto

\begin{enumerate}
    \item \textit{Potere di mercato}: la capacitá di tenere prezzi più alti del prezzo di concorrenza perfetta. Forte riduzione del surplus da parte del consumatore
    \item \textit{Presenta di esternalitá}: cioe degli effetti che vanno a riverberarsi su tutti gli altri agenti del mercato, effetto che non può essere compensato solo dal prezzo. Genera una disutilitá (i.e. inquinamento ambientale). É possibile cercare di compensare la disutilitá con un pagamento in denaro o di altro tipo, di fatto si genera un mercato ma in realtá è molto difficile valutare il danno. Nel caso dell'inquinamento le stime sono sempre più sofisticate ma rimane comunque difficile. L'esternalitá potrebbe anche essere positiva come gli \textit{spillover della ricerca}\footnote{Le informazioni della ricerca scientifica che vengono immediatamente rese pubbliche, in questo modo diventano esternalitá positive per le aziende e per l'intero sistema economico}. A livello economico comuque anche un'esternalità positiva non è detto che non crei un fallimento di mercato.
    \item \textit{Beni pubblici}: Tutti quei beni che sono:
        \begin{itemize}
            \item Non esclusivi: non è possibile escludere qualcuno dal loro utilizzo
            \item Non rivali: Il loro utilizzo non impedisce ad altri lo stesso utilizzo del bene
        \end{itemize}
    Si tratta di un fallimento di mercato perche avviene il fenomeno del \textit{Free riding}, questo genere di servizi viene finanziato da qualcuno in maniera maggiore rispetto ad altri, anche a paritá di utilizzo. Non conviene quindi alle imprese fornire quel tipo di servizio che viene fornito obbligatoriamente dallo stato. In questo contesto ansce il copyright.
    \item \textit{Asimmetria informativa}: Va ad attaccare l'impotesi n. 1 della concorrenza perfetta, ovvero la perfetta informazione. i.e. Quando agli inizi degli anni 2000 c'era il virus della mucca pazza la gente andava a comprare la carne senza saperne la provvenienza. Il mercato lasciato a se stesso è fallito, per correggere la situazinoe lo stato è intervenuto imponendo che tutti i venditori di carne dovessero tracciarne la provvenienza applicando un etichetta con su scritto l'indirizzo d'origine. A prova di questo, oggi non è possibile investire nel mercato di rischio tramite la banca senza aver compilato un foglio per l'analisi del profilo di rischio.
\end{enumerate}

\section{Monopolio}

Caratteristiche principali del monopolio:

\begin{itemize}
    \item 1 venditore, tanti compratori
    \item 1 prodotto (non ci sono beni sostituti)
    \item Barriere all'entrata
    \item Impresa price maker, è lei stessa a decidere il prezzo
\end{itemize}

Ricordiamo che i profitti (\(\pi\)) sono massimizzati a livello output se:

\begin{equation}\label{eq:MCMR}
    MR=MC
\end{equation}
\begin{conditions*}
    MR & Marginal Reveneu, ricavi marginali\\
    MC & Marginal Costs, costi marginali
\end{conditions*}

Il prezzo di equilibrio si ricava da:

\begin{equation}\label{eq:mrexploit}
    \begin{split}
        MR  &= P + P'(Q)*Q =\\
            &= P[1+P'(Q)*\frac{Q}{P}] =\\
            &= P+P(\frac{1}{E_D})=\\
            &=MC
    \end{split} 
\end{equation}

ricordando che:

\begin{equation}\label{eq:elasticita}
    \begin{split}
        E_D &= \frac{\frac{\Delta Q}{Q}}{\frac{\Delta P}{P}} = \frac{\Delta Q}{\Delta P} \frac{P}{Q} \\
        \lim_{\Delta P \to 0}E_D &= \frac{\partial Q}{\partial P}\frac{P}{Q}
    \end{split}
\end{equation}

ne deriva che la condizione di equilibrio, chiamato \textit{Indice di Lerner} è:

\begin{equation}\label{eq:indice di lerner}
    \frac{P-MC}{P}=\frac{1}{E_D}
\end{equation}

\begin{conditions*}
    P   &   	prezzo\\
    P'(Q)   &   Pendenza della retta della quantita, variazione del prezzo rispetto alla quantita\\
    Q   &   quantitá\\
    E_D &   Elasticita della domanda al prezzo, se \(> 1\) allora la domanda è elastica, cioe al variare del prezzo la domanda varia più che proporzionalmente, se \(<1\) ad una variazione del prezzo la domanda varia meno che proporzionalmente
\end{conditions*}

\begin{figure}[H]
    \centering
    \includegraphics[width=.7\linewidth]{images/chapter1/3.jpg}
    \caption{Curva della doppia condizione di mercato}
    \label{fig:curvaDoppiaCondizioneMercato}
\end{figure}

Dato che la condizione di equilibrio è la \ref{eq:MCMR}, prendo la quantitá corrispondente a quel punto e la proietto sulla curva della domanda (verticalmente), la distanza tra il prezzo e il costo marginale mi da il \textit{margine di guadagno}, più è incinata la domanda, quindi più è rigida \footnote{Una domanda perfettamente rigida verrebbe rappresentata con una curva verticale parallela all'asse delle ordinate}, più è alto il guadagno

\section{Potere di mercato}

La misura del \textit{Potere di mercato} in un mercato di monopolio è:

\begin{equation}\label{eq:lerner}
    L=\frac{P-MC}{P}
\end{equation}

\begin{conditions*}
    P-MC    &   Markup, differenza tra prezzo del bene e costo marginale\\
    P   &   Prezzo\\
    0<L<1   &   L è un valore compreso tra 0 e 1
\end{conditions*}

Più è alto \(L\) maggiore è il \textit{Potere di mercato}

\subsection{Fonti del potere di mercato}

Il potere di mercato è quindi funzione di diversi fattori, oltre all'\(E_D\), fattori che condizionano \(E_D\), condizionano anche indirettamente il potere di mercato:

\begin{enumerate}
    \item Elasticitá della domanda di mercato
    \item Numero di imprese nel mercato: barriere all'entrata e deterrenza di entrata
    \item Comportamento strategico delle imprese \textit{Incumbent}\footnote{Imprese che stanno entrando nel mercato}
    \item Nuove tecnologie
\end{enumerate}

\subsubsection{Elasticità della domanda di mercato}

Quando la domanda è anelastica il potere di mercato è maggiore, i.e. il petrolio ha domanda anelastica perchè è fondamentale per produrre energia, di cui non possiamo fare a meno. 

La presenza di \textit{fornitori alternativi} o prodotti sostituti riducono il potere di mercato, una soluzione all'anelasticità è quindi trovare alternative, trovando fornitori alternativi, nel medio periodo, la domanda si elasticizza.

Per lo stesso motivo i \textit{prodotti sostituti} rendono la domanda più elastica.

\subsubsection{Numero di imprese}

La concorrenza non dipende dal numero di imprese in via esclusiva, dipende anche dalla strategia adottata da esse (i.e. dipende se le imprese competono sul prezzo o sulle quantita).

Il numero di imprese di per se è poco significativo, occorre utilizzare infatti la quantitá di mercato che ogni impresa possiede, un mercato è più concentrato se ci sono solo poche imprese che detengono la maggior parte della quota di mercato (i.e. Google etc.).

Le imprese presenti generano delle barriere all'entrata che possono essere di diversi tipi:
\begin{itemize}
    \item Brevetti (patents)
    \item Copyrights
    \item Licenze
    \item Economie di scala
\end{itemize}

\subsubsection{Economie di scala}

Ci sono essenzialmente due tipologie di barriere all'entrata:
\begin{enumerate}
    \item Di tipo \textit{Strutturale}: quando dipendono dalla tecnologia del settore industriale:
    \begin{itemize}
        \item \textit{Economie di scala}: permettono di ridurre i costi medi di produzione all'autore della produzione
        \begin{figure}[H]
            \centering
            \includegraphics[width=.7\linewidth]{images/chapter1/4.jpg}
            \caption{Curva dei costi medi in presenza di economie di scala}
            \label{fig:economiediscala}
        \end{figure}
        Se da \(\overline{q}\) arrivo a \(q^*\) riesco ad aumentare la quantità prodotta (\(q\)), aumentando anche l'efficienza (diminuendo i costi). La caratteristica di questo contesto è che risulta più efficiente una azienda molto più grande rispetto a due aziende medie. Se ho due imprese che producono a \(\frac{q^*}{2}\) avró un costo medio di fornitura pari ad \(AC(\frac{q^*}{2})*2 = AC(q^*)\), molto più elevato in proporzione rispetto al caso precedente. Ne consegue che l'ipresa piccola che deve entrare in un mercato con queste caratteristiche andrá in contro ad una serie di costi con l'andamento in figura \ref{fig:economiediscala} con dei costi medi molto alti \(AC(\widetilde(q))\).
        
        Si hanno dei costi con questo andamento nei settori \textit{Capital Intensive}, quei settori che hanno bisogno di investimenti altissimi per poter iniziare a produrre all'interno dei quali i costi fissi sono molto alti.

        Tutti i settori \textit{Capital Intesive} hanno barriere all'entrata molto alte.

        \item \textit{Costi sunk}: quando un'impresa incumbent deve sostenere costi per il semplice fatto di entrare nel mercato, costi che quelli giá all'interno non devono sostenere, i costi talvolta sono così alti da affossare l'azienda fin da subito.
        \item \textit{Vantaggi di costo assoluto}: Il ptoere contrattuale delle aziende incumbent può essere molto minore delle aziende giá presenti nel mercato, le quali possono strappare offerte e prezzi decisamente vantaggiosi
        \item \textit{Costi sunk da parte dei consumatori}: se un consumatore ha un enorme \textit{switching cost}sicuramente non cambierá il prodotto che giá sta utilizzando. In molti contesti la libertá  di movimento viene meno (i.e. portabilitá del numero nei primi anni 2000, se un individuo desiderava cambiare operatore era costretto a cambiare numero).
        Il \textit{Switching cost} potrebbe anche essere un costo vero e proprio, da contratto sottoforma di penale.

    \end{itemize}
\end{enumerate}






