\chapter{Concorrenza perfetta, monopolio, potere e fallimenti di mercato}
\label{sec:Concorrenza perfetta, monopolio, potere e fallimenti di mercato}

\section{Concorrenza perfetta}
\begin{itemize}
    \item Aziende Price Taker: le imprese non possono fissare il prezzo, vendono al prezzo di mercato
    \item Il prezzo deriva dall'interazione tra domanda e offerta, l'unico in grado di fissarlo è quindi il mercato
    \item I prodotti sono omogenei
    \item L'informazione è perfetta
\end{itemize}

\begin{equation} \label{eq:1}
    p=(L)ML=(LR)AC
\end{equation}

Come vediamo nell'equazione \ref{eq:1} i prezzi vengono fissati ai costi marginali incrementali, nel lungo periodo si dimostra che il prezzo è anche pari al costo medio e di conseguenza nel lungo periodo le imprese non hanno extra-profitti.

Durante il corso si considererá solamente la forma di mercato "concorrenza perfetta" perché è l'unica che massimizza il benessere collettivo (welfare). Favorire la concorrenza è quindi importante per avvicinarsi allo stato di benessere collettivo ideale.

Nell'immagine \ref{fig:curva_domanda_offerta} è possibile osservare la curva di domanda (decrescente) e la curva dell'offerta (crescente). Aumentando la domanda, spostando quindi la curva decrescente a destra,  li prezzo di equilibrio definito da \(p*\) si sposterebbe lungo la curva dell'offerta verso destra aumentando. Viceversa aumentando l'offerta il prezzo \(p*\) si sposterebbe lungo la curva della domanda diminuendo.
\begin{figure}[h]
    \centering
    \includegraphics[width=.7\linewidth]{images/chapter1/1.jpg}
    \caption{Curva di domanda e offerta}
    \label{fig:curva_domanda_offerta}
\end{figure}

\section{Domanda individuale: teoria del consumatore}

\begin{equation}
    \label{eq:2}
    \begin{gathered}
        \nu(p,m)=max(u(x)) \\
        s.t. \underbrace{p*q = m}_{\makebox[0pt]{Vincolo di bilancio}}
    \end{gathered}
\end{equation}
\begin{conditions*}
    p*q &   spesa del consumatore \\
    m   &   reddito
\end{conditions*}

Il vincolo di bilancio quindi tiene conto di reddito e prezzi, risolvendo l'equazione trovo \(q\) cioè la quantità. Trovando il massimo della funzione di utilità rispetto alla quantità trovo \(q'\).

Il problema di massimizzazione dell'utilità può essere risolto utilizzando la Lagrangiana:
\begin{equation}
    \label{eq:lagrange}
    L=u(q)-\lambda(p*q-m)
\end{equation}
\begin{conditions*}
    u(q)    &   Spesa del consumatore \\
    \lambda &   Reddito \\
    p*q-m   &   Vincolo di bilancio
\end{conditions*}

A questo punto si calcola la derivata dell'utilità e la si eguaglia a 0 per massimizzarla:
\begin{equation}
    \frac{\partial u(q)}{\partial q_i} - \lambda(p_i) = 0
\end{equation}

Che può essere riscritta come segue
\begin{equation}
    \underbrace{\frac{\frac{\partial u(q*)}{\partial q_i}}{\frac{\partial u(q*)}{\partial q_j}}}_{(a)} = \underbrace{\frac{p_i}{p_j}}_{(b)}
\end{equation}
\begin{conditions*}
    (a) &   saggio marginale di sostituzione, corrisponde all'inclinazione della curva di indifferenza \\
    (b) &   rapporto tra i prezzi che corrisponde all'inclinazione della curva di bilancio
\end{conditions*}

Questa condizione nasconde il rapporto tra effetto reddito ed effetto sostituzione, un questo corso si presuppone la condizione di \textbf{equilibrio parziale}.

\subsection{Funzione di utilità quasi-lineare}

Si considera un mercato preso singolarmente, idealmente isolato rispetto agli altri, in questo contesto \textbf{equilibrio parziale} significa che una variazione all'interno di un mercato impatta principalmente sul mercato stesso e non nell'economia globale.

Si introducono quindi 2 ipotesi semplificative:
\begin{enumerate}
    \item \label{itm:1}Effetto sostituzione nullo
    \item \label{itm:2}Effetto reddito nullo
\end{enumerate}

Il prezzo degli altri beni può essere considerato fisso e può essere normalizzato a "1". È possibile quindi , a fronte delle condizioni \ref{itm:1} and \ref{itm:2} riscrivere la funzione di utilità come segue:

\begin{equation}
    \label{eq:3}
    \begin{gathered}
        U(x,y)=u(x)+y \\
        s.t. p*x + y = m
    \end{gathered}
\end{equation}
\begin{conditions*}
    p*x &   spesa del consumatore \\
    y   &   quantità del bene y \\
    m   &   reddito
\end{conditions*}

Dopo aver calcolato la lagrangiana, come nell'equazione \ref{eq:lagrange}, massimizzo nuovamente l'utilità, sia in termini di x che in termini di y:

\begin{equation}
    \label{eq:4}
    \begin{split}
        &\frac{\partial L}{\partial x} = \frac{\partial u(x)}{\partial x} - \lambda*p = 0 \\
        &\frac{\partial L}{\partial y} = 1- \lambda = 0
    \end{split}
\end{equation}

Unendo le equazioni \ref{eq:4} ottengo:

\begin{equation}\label{eq:maxutilita}
    \frac{\partial u(x)}{\partial x} = u'(x) = p
\end{equation}
\begin{conditions*}
    u'(x)   &   utilità marginale \\
    p       &   prezzo
\end{conditions*}

Ne consegue che l'utilità marginale è uguale al prezzo.

In queste condizioni quindi la quantità \(x\) dipende solo e soltanto dal prezzo, non si ha pertanto effetto reddito.

L'interpretazione è quindi molto più diretta, il consumatore comprerà il bene fino a quando la sua utilità marginale sarà maggiore o uguale al prezzo.

\subsection{Surplus}

\begin{itemize}
    \item Surplus del consumatore: beneficio totale o valore che il consumatore riceve oltre al prezzo pagato.
    \item Surplus del produttore: beneficio totale o ricavo/profitto ch eun produttore riceve oltre ai costi di produzione
\end{itemize}

\begin{figure}[H]
    \centering
    \includegraphics[width=.6\linewidth]{images/chapter1/2.jpg}
    \caption{Surplus del consumatore (in giallo) e surplus del produttore (in blu)}
    \label{fig:curva_surplus}
\end{figure}

Tutte le aziende vendono un prodotto ad un prezzo uguale pari a 5.

\begin{itemize}
    \item I consumatore nel punto \(Q_d\): massimizzerebbero la loro utilità anche pagando 9, sarebbero quindi disposti a pagare 9 ma pagano 5, hanno quindi un surplus (del consumatore) pari a 4.
    \item I produttore in \(Q_s\) massimizzerebbero la loro utilià anche se vendessero ad un prezzo pari a 3, ricavano però 5 quindi hanno un surplus di 2.
\end{itemize}

È molto iportante capire come molte delle decisioni dipendono da questo surplus, sia dal lato del consumatore che dal lato del produttore.

Possiamo calcolare l'area dei due triangoli in due modi differenti:

\begin{enumerate}
    \item Dal lato del prezzo (guardando il grafico orizzontalmente)

          \begin{equation}
              CS = V(p) = \int_{p*}^{\infty}q(p)\,dp
          \end{equation}

          risulta quindi:
          \begin{equation}
              \frac{\partial V(p)}{\partial p} = -q(p)
          \end{equation}

          La domanda presenta quindi un'inclinazione negativa, il segno "-" è dovuto al fatto che \(p*\) è l'estremo inferiore. Se aumenta il prezzo, il surplus del consumatore diminuisce.

    \item Dal lato della quantitá (guardando il grafico verticalmente):
          \begin{equation}
              CS = S(q) - p(q)*q
          \end{equation}

          dove:
          \begin{equation}
              S(q) = \int_{0}^{q*}  p(q)\,dq
          \end{equation}
          \(S(q)\) è quindi il surplus lordo, ovvero il trapezio rettangolo che comprende il triangolo giallo, il triangolo blu e la base maggiore fino a \(Q_0\).

          Risulta quindi:
          \begin{equation}
              \frac{\partial S(q)}{\partial q} = p(q)
          \end{equation}

          \begin{conditions*}
              p(q)            &   spesa del consumatore\\
              \partial S(q)   &   Variazione del surplus lordo
          \end{conditions*}

          In questo caso se aumento la quantità l'area del surplus lordo aumenta.
\end{enumerate}

Ne consegue che il surplus del consumatore ha due definizioni differenti a seconda della variabile di riferimento utilizzata (prezzo o quantità).

\section{Economia del Benessere (Welfare)}
\subsection{Applicazioni in concorrenza perfetta}
Che effetto ha la presenza di un mercato in concorrenza perfetta sull'utilitá del consumatore?
\begin{itemize}
    \item 1 impresa in concorrenza perfetta
    \item 1 consuamatore
\end{itemize}
l'utilitá massimizzata è l'equazione \ref{eq:maxutilita} quindi:
\[u'(x)=p\]
L'impresa in questione ha un costo \(c'>0\), \(c''>0\) e \(c(0)=0\), in concorrenza perfetta quindi
\[p=c'(x)\]
Ne consegue che:
\begin{equation}
    u'(x)=c'(x)
\end{equation}
Questa condizione mi diche che il consumatore continuerà a consumare il bene fino a quando il beneficio che ottiene coprirá i costi di produzione di quel bene.

\subsection{Applicazione con funzione di Welfare}

Definiamo quindi la funzione di Welfare:
\begin{equation}
    \label{eq:welfare}
    \begin{split}
        W   &= max[CS(q) + PS(q)] =\\
            &= [u(q) - p(q)] + p*q - c*q =\\
            &= u(q) - c(q)
    \end{split}
\end{equation}
\begin{conditions*}
    q & quantitá
\end{conditions*}
Con lo sviluppo \ref{eq:welfare} si dimostra quindi che la concorrenza perfetta massimizza l'utilitá del consumatore e il benessere collettivo. È quindi considerata il caso \textbf{benchmark}.

\subsection{Primo teorema dell'economia del benessere}

\begin{theorem}
    Se tutti i mercati fossero in concorrenza perfetta ogni scambio porterebbe alla migliore allocazione delle risorse possibile e sarebbero, di conseguenza, economicamente efficienti.
\end{theorem}

\begin{theorem}
    Ogni allontanamento dalle condizioni di concorrenza perfetta ha, come conseguenza, un effetto peggiorativo sul benessere collettivo.\footnote{Un peggioramento del benessere collettivo può significare anche uno sbilanciamento come ad esempio un aumento del benessere delle sole imprese a scapito di una diminuzione del benessere dei consumatori o viceveversa}
\end{theorem}

\subsection{Equitá ed Efficienza}
L'equilibrio dei mercati porta anche ad un allocazione delle risorse equa a livello sociale?

La risposta è \textbf{no}, non c'è nessuna ragione per dire che un mercato efficiente porta ad un'allocazione delle risorse socialmente equa, da qui deriva il secondo teorema del benessere;

\subsection{Secondo teorema dell'economia del benessere}

\begin{theorem}
    La soluzione dell'economia perfetta porterebbe ad una soluzione equa dal punto di vista sociale solo sotto la condizione stringente che cci sia una riallocazione iniziale dei beni
\end{theorem}

Es. un ricco, prima di usufruire di un bene, dovrebbe distribuire i suoi averi alla collettività, in modo che poi, secondo le regole della concorrenza perfetta, tutti potrebbero usufruire in modo uguale di quello stesso bene.

Una soluzione parziale e incompleta è quella della tassazione.

\section{Fallimenti di mercato}

Rappresentano tutte le cause che rendono il mercato imperfetto

\begin{enumerate}
    \item \textit{Potere di mercato}: la capacitá di tenere prezzi più alti del prezzo di concorrenza perfetta. Forte riduzione del surplus da parte del consumatore
    \item \textit{Presenta di esternalitá}: cioe degli effetti che vanno a riverberarsi su tutti gli altri agenti del mercato, effetto che non può essere compensato solo dal prezzo. Genera una disutilitá (i.e. inquinamento ambientale)
\end{enumerate}





