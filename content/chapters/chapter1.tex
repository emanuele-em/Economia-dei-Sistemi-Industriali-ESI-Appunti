\chapter{Concorrenza perfetta, monopolio, potere e fallimenti di mercato}
\label{sec:Concorrenza perfetta, monopolio, potere e fallimenti di mercato}

\section{Concorrenza perfetta}
\begin{itemize}
    \item Aziende Price Taker: le imprese non possono fissare il prezzo, vendono al prezzo di mercato
    \item Il prezzo deriva dall'interazione tra domanda e offerta, l'unico in grado di fissarlo è quindi il mercato
    \item I prodotti sono omogenei
    \item L'informazione è perfetta
\end{itemize}

\begin{equation} \label{eq:1}
    p=(L)ML=(LR)AC
\end{equation}

Come vediamo nell'equazione \ref{eq:1} i prezzi vengono fissati ai costi marginali incrementali, nel lungo periodo si dimostra che il prezzo è anche pari al costo medio e di conseguenza nel lungo periodo le imprese non hanno extra-profitti.

Durante il corso si considererá solamente la forma di mercato "concorrenza perfetta" perché è l'unica che massimizza il benessere collettivo (welfare). Favorire la concorrenza è quindi importante per avvicinarsi allo stato di benessere collettivo ideale.

Nell'immagine \ref{fig:curva_domanda_offerta} è possibile osservare la curva di domanda (decrescente) e la curva dell'offerta (crescente). Aumentando la domanda, spostando quindi la curva decrescente a destra,  li prezzo di equilibrio definito da \(p*\) si sposterebbe lungo la curva dell'offerta verso destra aumentando. Viceversa aumentando l'offerta il prezzo \(p*\) si sposterebbe lungo la curva della domanda diminuendo.
\begin{figure}[H]
    \centering
    \includegraphics[width=.7\linewidth]{images/chapter1/1.jpg}
    \caption{Curva di domanda e offerta}
    \label{fig:curva_domanda_offerta}
\end{figure}

\section{Domanda individuale: teoria del consumatore}

\begin{equation}
    \label{eq:2}
    \begin{gathered}
        \nu(p,m)=max(u(x)) \\
        s.t. \underbrace{p*q = m}_{\makebox[0pt]{Vincolo di bilancio}}
    \end{gathered}
\end{equation}
\begin{conditions*}
    p*q &   spesa del consumatore \\
    m   &   reddito
\end{conditions*}

Il vincolo di bilancio quindi tiene conto di reddito e prezzi, risolvendo l'equazione trovo \(q\) cioè la quantità. Trovando il massimo della funzione di utilità rispetto alla quantità trovo \(q'\).

Il problema di massimizzazione dell'utilità può essere risolto utilizzando la Lagrangiana:
\begin{equation}
    \label{eq:lagrange}
    L=u(q)-\lambda(p*q-m)
\end{equation}
\begin{conditions*}
    u(q)    &   Spesa del consumatore \\
    \lambda &   Reddito \\
    p*q-m   &   Vincolo di bilancio
\end{conditions*}

A questo punto si calcola la derivata dell'utilità e la si eguaglia a 0 per massimizzarla:
\begin{equation}
    \frac{\partial u(q)}{\partial q_i} - \lambda(p_i) = 0
\end{equation}

Che può essere riscritta come segue
\begin{equation}
    \underbrace{\frac{\frac{\partial u(q*)}{\partial q_i}}{\frac{\partial u(q*)}{\partial q_j}}}_{(a)} = \underbrace{\frac{p_i}{p_j}}_{(b)}
\end{equation}
\begin{conditions*}
    (a) &   saggio marginale di sostituzione, corrisponde all'inclinazione della curva di indifferenza \\
    (b) &   rapporto tra i prezzi che corrisponde all'inclinazione della curva di bilancio
\end{conditions*}

Questa condizione nasconde il rapporto tra effetto reddito ed effetto sostituzione, un questo corso si presuppone la condizione di \textbf{equilibrio parziale}.

\subsection{Funzione di utilità quasi-lineare}

Si considera un mercato preso singolarmente, idealmente isolato rispetto agli altri, in questo contesto \textbf{equilibrio parziale} significa che una variazione all'interno di un mercato impatta principalmente sul mercato stesso e non nell'economia globale.

Si introducono quindi 2 ipotesi semplificative:
\begin{enumerate}
    \item \label{itm:1}Effetto sostituzione nullo
    \item \label{itm:2}Effetto reddito nullo
\end{enumerate}

Il prezzo degli altri beni può essere considerato fisso e può essere normalizzato a "1". È possibile quindi , a fronte delle condizioni \ref{itm:1} and \ref{itm:2} riscrivere la funzione di utilità come segue:

\begin{equation}
    \label{eq:3}
    \begin{gathered}
        U(x,y)=u(x)+y \\
        s.t. p*x + y = m
    \end{gathered}
\end{equation}
\begin{conditions*}
    p*x &   spesa del consumatore \\
    y   &   quantità del bene y \\
    m   &   reddito
\end{conditions*}

Dopo aver calcolato la lagrangiana, come nell'equazione \ref{eq:lagrange}, massimizzo nuovamente l'utilità, sia in termini di x che in termini di y:

\begin{equation}
    \label{eq:4}
    \begin{split}
        &\frac{\partial L}{\partial x} = \frac{\partial u(x)}{\partial x} - \lambda*p = 0 \\
        &\frac{\partial L}{\partial y} = 1- \lambda = 0
    \end{split}
\end{equation}

Unendo le equazioni \ref{eq:4} ottengo:

\begin{equation}\label{eq:maxutilita}
    \frac{\partial u(x)}{\partial x} = u'(x) = p
\end{equation}
\begin{conditions*}
    u'(x)   &   utilità marginale \\
    p       &   prezzo
\end{conditions*}

Ne consegue che l'utilità marginale è uguale al prezzo.

In queste condizioni quindi la quantità \(x\) dipende solo e soltanto dal prezzo, non si ha pertanto effetto reddito.

L'interpretazione è quindi molto più diretta, il consumatore comprerà il bene fino a quando la sua utilità marginale sarà maggiore o uguale al prezzo.

\subsection{Surplus}

\begin{itemize}
    \item Surplus del consumatore: beneficio totale o valore che il consumatore riceve oltre al prezzo pagato.
    \item Surplus del produttore: beneficio totale o ricavo/profitto ch eun produttore riceve oltre ai costi di produzione
\end{itemize}

\begin{figure}[H]
    \centering
    \includegraphics[width=.6\linewidth]{images/chapter1/2.jpg}
    \caption{Surplus del consumatore (in giallo) e surplus del produttore (in blu)}
    \label{fig:curva_surplus}
\end{figure}

Tutte le aziende vendono un prodotto ad un prezzo uguale pari a 5.

\begin{itemize}
    \item I consumatore nel punto \(Q_d\): massimizzerebbero la loro utilità anche pagando 9, sarebbero quindi disposti a pagare 9 ma pagano 5, hanno quindi un surplus (del consumatore) pari a 4.
    \item I produttore in \(Q_s\) massimizzerebbero la loro utilià anche se vendessero ad un prezzo pari a 3, ricavano però 5 quindi hanno un surplus di 2.
\end{itemize}

È molto iportante capire come molte delle decisioni dipendono da questo surplus, sia dal lato del consumatore che dal lato del produttore.

Possiamo calcolare l'area dei due triangoli in due modi differenti:

\begin{enumerate}
    \item Dal lato del prezzo (guardando il grafico orizzontalmente)

          \begin{equation}
              CS = V(p) = \int_{p*}^{\infty}q(p)\,dp
          \end{equation}

          risulta quindi:
          \begin{equation}
              \frac{\partial V(p)}{\partial p} = -q(p)
          \end{equation}

          La domanda presenta quindi un'inclinazione negativa, il segno "-" è dovuto al fatto che \(p*\) è l'estremo inferiore. Se aumenta il prezzo, il surplus del consumatore diminuisce.

    \item Dal lato della quantitá (guardando il grafico verticalmente):
          \begin{equation}
              CS = S(q) - p(q)*q
          \end{equation}

          dove:
          \begin{equation}
              S(q) = \int_{0}^{q*}  p(q)\,dq
          \end{equation}
          \(S(q)\) è quindi il surplus lordo, ovvero il trapezio rettangolo che comprende il triangolo giallo, il triangolo blu e la base maggiore fino a \(Q_0\).

          Risulta quindi:
          \begin{equation}
              \frac{\partial S(q)}{\partial q} = p(q)
          \end{equation}

          \begin{conditions*}
              p(q)            &   spesa del consumatore\\
              \partial S(q)   &   Variazione del surplus lordo
          \end{conditions*}

          In questo caso se aumento la quantità l'area del surplus lordo aumenta.
\end{enumerate}

Ne consegue che il surplus del consumatore ha due definizioni differenti a seconda della variabile di riferimento utilizzata (prezzo o quantità).

\section{Economia del Benessere (Welfare)}
\subsection{Applicazioni in concorrenza perfetta}
Che effetto ha la presenza di un mercato in concorrenza perfetta sull'utilitá del consumatore?
\begin{itemize}
    \item 1 impresa in concorrenza perfetta
    \item 1 consuamatore
\end{itemize}
l'utilitá massimizzata è l'equazione \ref{eq:maxutilita} quindi:
\[u'(x)=p\]
L'impresa in questione ha un costo \(c'>0\), \(c''>0\) e \(c(0)=0\), in concorrenza perfetta quindi
\[p=c'(x)\]
Ne consegue che:
\begin{equation}
    u'(x)=c'(x)
\end{equation}
Questa condizione mi diche che il consumatore continuerà a consumare il bene fino a quando il beneficio che ottiene coprirá i costi di produzione di quel bene.

\subsection{Applicazione con funzione di Welfare}

Definiamo quindi la funzione di Welfare:
\begin{equation}
    \label{eq:welfare}
    \begin{split}
        W   &= max[CS(q) + PS(q)] =\\
            &= [u(q) - p(q)] + p*q - c*q =\\
            &= u(q) - c(q)
    \end{split}
\end{equation}
\begin{conditions*}
    q & quantitá
\end{conditions*}
Con lo sviluppo \ref{eq:welfare} si dimostra quindi che la concorrenza perfetta massimizza l'utilitá del consumatore e il benessere collettivo. È quindi considerata il caso \textbf{benchmark}.

\subsection{Primo teorema dell'economia del benessere}

\begin{theorem}
    Se tutti i mercati fossero in concorrenza perfetta ogni scambio porterebbe alla migliore allocazione delle risorse possibile e sarebbero, di conseguenza, economicamente efficienti.
\end{theorem}

\begin{theorem}
    Ogni allontanamento dalle condizioni di concorrenza perfetta ha, come conseguenza, un effetto peggiorativo sul benessere collettivo.\footnote{Un peggioramento del benessere collettivo può significare anche uno sbilanciamento come ad esempio un aumento del benessere delle sole imprese a scapito di una diminuzione del benessere dei consumatori o viceveversa}
\end{theorem}

\subsection{Equitá ed Efficienza}
L'equilibrio dei mercati porta anche ad un allocazione delle risorse equa a livello sociale?

La risposta è \textbf{no}, non c'è nessuna ragione per dire che un mercato efficiente porta ad un'allocazione delle risorse socialmente equa, da qui deriva il secondo teorema del benessere;

\subsection{Secondo teorema dell'economia del benessere}

\begin{theorem}
    La soluzione dell'economia perfetta porterebbe ad una soluzione equa dal punto di vista sociale solo sotto la condizione stringente che cci sia una riallocazione iniziale dei beni
\end{theorem}

Es. un ricco, prima di usufruire di un bene, dovrebbe distribuire i suoi averi alla collettività, in modo che poi, secondo le regole della concorrenza perfetta, tutti potrebbero usufruire in modo uguale di quello stesso bene.

Una soluzione parziale e incompleta è quella della tassazione.

\section{Fallimenti di mercato}

Rappresentano tutte le cause che rendono il mercato imperfetto

\begin{enumerate}
    \item \textit{Potere di mercato}: la capacitá di tenere prezzi più alti del prezzo di concorrenza perfetta. Forte riduzione del surplus da parte del consumatore
    \item \textit{Presenta di esternalitá}: cioe degli effetti che vanno a riverberarsi su tutti gli altri agenti del mercato, effetto che non può essere compensato solo dal prezzo. Genera una disutilitá (i.e. inquinamento ambientale). É possibile cercare di compensare la disutilitá con un pagamento in denaro o di altro tipo, di fatto si genera un mercato ma in realtá è molto difficile valutare il danno. Nel caso dell'inquinamento le stime sono sempre più sofisticate ma rimane comunque difficile. L'esternalitá potrebbe anche essere positiva come gli \textit{spillover della ricerca}\footnote{Le informazioni della ricerca scientifica che vengono immediatamente rese pubbliche, in questo modo diventano esternalitá positive per le aziende e per l'intero sistema economico}. A livello economico comuque anche un'esternalità positiva non è detto che non crei un fallimento di mercato.
    \item \textit{Beni pubblici}: Tutti quei beni che sono:
        \begin{itemize}
            \item Non esclusivi: non è possibile escludere qualcuno dal loro utilizzo
            \item Non rivali: Il loro utilizzo non impedisce ad altri lo stesso utilizzo del bene
        \end{itemize}
    Si tratta di un fallimento di mercato perche avviene il fenomeno del \textit{Free riding}, questo genere di servizi viene finanziato da qualcuno in maniera maggiore rispetto ad altri, anche a paritá di utilizzo. Non conviene quindi alle imprese fornire quel tipo di servizio che viene fornito obbligatoriamente dallo stato. In questo contesto ansce il copyright.
    \item \textit{Asimmetria informativa}: Va ad attaccare l'impotesi n. 1 della concorrenza perfetta, ovvero la perfetta informazione. i.e. Quando agli inizi degli anni 2000 c'era il virus della mucca pazza la gente andava a comprare la carne senza saperne la provvenienza. Il mercato lasciato a se stesso è fallito, per correggere la situazinoe lo stato è intervenuto imponendo che tutti i venditori di carne dovessero tracciarne la provvenienza applicando un etichetta con su scritto l'indirizzo d'origine. A prova di questo, oggi non è possibile investire nel mercato di rischio tramite la banca senza aver compilato un foglio per l'analisi del profilo di rischio.
\end{enumerate}

\section{Monopolio}

Caratteristiche principali del monopolio:

\begin{itemize}
    \item 1 venditore, tanti compratori
    \item 1 prodotto (non ci sono beni sostituti)
    \item Barriere all'entrata
    \item Impresa price maker, è lei stessa a decidere il prezzo
\end{itemize}

Ricordiamo che i profitti (\(\pi\)) sono massimizzati a livello output se:

\begin{equation}\label{eq:MCMR}
    MR=MC
\end{equation}
\begin{conditions*}
    MR & Marginal Reveneu, ricavi marginali\\
    MC & Marginal Costs, costi marginali
\end{conditions*}

Il prezzo di equilibrio si ricava da:

\begin{equation}\label{eq:mrexploit}
    \begin{split}
        MR  &= P + P'(Q)*Q =\\
            &= P[1+P'(Q)*\frac{Q}{P}] =\\
            &= P+P(\frac{1}{E_D})=\\
            &=MC
    \end{split} 
\end{equation}

ricordando che:

\begin{equation}\label{eq:elasticita}
    \begin{split}
        E_D &= \frac{\frac{\Delta Q}{Q}}{\frac{\Delta P}{P}} = \frac{\Delta Q}{\Delta P} \frac{P}{Q} \\
        \lim_{\Delta P \to 0}E_D &= \frac{\partial Q}{\partial P}\frac{P}{Q}
    \end{split}
\end{equation}

ne deriva che la condizione di equilibrio, chiamato \textit{Indice di Lerner} è:

\begin{equation}\label{eq:indice di lerner}
    \frac{P-MC}{P}=\frac{1}{E_D}
\end{equation}

\begin{conditions*}
    P   &   	prezzo\\
    P'(Q)   &   Pendenza della retta della quantita, variazione del prezzo rispetto alla quantita\\
    Q   &   quantitá\\
    E_D &   Elasticita della domanda al prezzo, se \(> 1\) allora la domanda è elastica, cioe al variare del prezzo la domanda varia più che proporzionalmente, se \(<1\) ad una variazione del prezzo la domanda varia meno che proporzionalmente
\end{conditions*}

\begin{figure}[H]
    \centering
    \includegraphics[width=.7\linewidth]{images/chapter1/3.jpg}
    \caption{Curva della doppia condizione di mercato}
    \label{fig:curvaDoppiaCondizioneMercato}
\end{figure}

Dato che la condizione di equilibrio è la \ref{eq:MCMR}, prendo la quantitá corrispondente a quel punto e la proietto sulla curva della domanda (verticalmente), la distanza tra il prezzo e il costo marginale mi da il \textit{margine di guadagno}, più è incinata la domanda, quindi più è rigida \footnote{Una domanda perfettamente rigida verrebbe rappresentata con una curva verticale parallela all'asse delle ordinate}, più è alto il guadagno

\section{Potere di mercato}

La misura del \textit{Potere di mercato} in un mercato di monopolio è:

\begin{equation}\label{eq:lerner}
    L=\frac{1}{E_D}=\frac{P-MC}{P}
\end{equation}

\begin{conditions*}
    P-MC    &   Markup, differenza tra prezzo del bene e costo marginale\\
    P   &   Prezzo\\
    0<L<1   &   L è un valore compreso tra 0 e 1
\end{conditions*}

Più è alto \(L\) maggiore è il \textit{Potere di mercato}

\subsection{Fonti del potere di mercato}

Il potere di mercato è quindi funzione di diversi fattori, oltre all'\(E_D\), fattori che condizionano \(E_D\), condizionano anche indirettamente il potere di mercato:

\begin{enumerate}
    \item Elasticitá della domanda di mercato
    \item Numero di imprese nel mercato: barriere all'entrata e deterrenza di entrata
    \item Comportamento strategico delle imprese \textit{Incumbent}\footnote{Imprese che stanno entrando nel mercato}
    \item Nuove tecnologie
\end{enumerate}

\subsubsection{Elasticità della domanda di mercato}

Quando la domanda è anelastica il potere di mercato è maggiore, i.e. il petrolio ha domanda anelastica perchè è fondamentale per produrre energia, di cui non possiamo fare a meno. 

La presenza di \textit{fornitori alternativi} o prodotti sostituti riducono il potere di mercato, una soluzione all'anelasticità è quindi trovare alternative, trovando fornitori alternativi, nel medio periodo, la domanda si elasticizza.

Per lo stesso motivo i \textit{prodotti sostituti} rendono la domanda più elastica.

\subsubsection{Numero di imprese}

La concorrenza non dipende dal numero di imprese in via esclusiva, dipende anche dalla strategia adottata da esse (i.e. dipende se le imprese competono sul prezzo o sulle quantita).

Il numero di imprese di per se è poco significativo, occorre utilizzare infatti la quantitá di mercato che ogni impresa possiede, un mercato è più concentrato se ci sono solo poche imprese che detengono la maggior parte della quota di mercato (i.e. Google etc.).

Le imprese presenti generano delle barriere all'entrata che possono essere di diversi tipi:
\begin{itemize}
    \item Brevetti (patents)
    \item Copyrights
    \item Licenze
    \item Economie di scala
\end{itemize}

\subsubsection{Economie di scala}

Ci sono essenzialmente due tipologie di barriere all'entrata:
\begin{enumerate}
    \item Di tipo \textit{Strutturale}: quando dipendono dalla tecnologia del settore industriale:
        \subitem \textit{Economie di scala}: permettono di ridurre i costi medi di produzione all'autore della produzione
        \begin{figure}[H]
            \centering
            \includegraphics[width=.7\linewidth]{images/chapter1/4.jpg}
            \caption{Curva dei costi medi in presenza di economie di scala}
            \label{fig:economiediscala}
        \end{figure}
        Se da \(\overline{q}\) arrivo a \(q^*\) riesco ad aumentare la quantità prodotta (\(q\)), aumentando anche l'efficienza (diminuendo i costi). La caratteristica di questo contesto è che risulta più efficiente una azienda molto più grande rispetto a due aziende medie. Se ho due imprese che producono a \(\frac{q^*}{2}\) avró un costo medio di fornitura pari ad \(AC(\frac{q^*}{2})*2 = AC(q^*)\), molto più elevato in proporzione rispetto al caso precedente. Ne consegue che l'ipresa piccola che deve entrare in un mercato con queste caratteristiche andrá in contro ad una serie di costi con l'andamento in figura \ref{fig:economiediscala} con dei costi medi molto alti \(AC(\widetilde(q))\).
        
        Si hanno dei costi con questo andamento nei settori \textit{Capital Intensive}, quei settori che hanno bisogno di investimenti altissimi per poter iniziare a produrre all'interno dei quali i costi fissi sono molto alti.

        Tutti i settori \textit{Capital Intesive} hanno barriere all'entrata molto alte.

        \subitem \textit{Costi sunk}: quando un'impresa incumbent deve sostenere costi per il semplice fatto di entrare nel mercato, costi che quelli giá all'interno non devono sostenere, i costi talvolta sono così alti da affossare l'azienda fin da subito.
        \subitem \textit{Vantaggi di costo assoluto}: Il ptoere contrattuale delle aziende incumbent può essere molto minore delle aziende giá presenti nel mercato, le quali possono strappare offerte e prezzi decisamente vantaggiosi
        \subitem \textit{Costi sunk da parte dei consumatori}: se un consumatore ha un enorme \textit{switching cost}sicuramente non cambierá il prodotto che giá sta utilizzando. In molti contesti la libertá  di movimento viene meno (i.e. portabilitá del numero nei primi anni 2000, se un individuo desiderava cambiare operatore era costretto a cambiare numero).
        Il \textit{Switching cost} potrebbe anche essere un costo vero e proprio, da contratto sottoforma di penale.
    \item Di tipo \textit{comportaentale o strategico}: cioe la deterrenza all'entrata quando le imprese che sono dentro al mercato cercano di impoedire con delle strategie l'entrata di nuove imprese
        \subitem \textit{Comportamento aggressivo post entrata}: i.e. l'acciaio è un settore molto capital intensive, in America fu scomperto un modo per produrre acciaio a basso costo, ALCOA, leader nel settore, allora comprò tutti gli impianti di produzione americani e ne costrui di nuovi lasciandoli peró fermi. La strategia consisteva nel far partire la produzione nel momento in cui il concorrente nuovo e tecnologicamente avanzato sarebbe entrato, in questo modo avrebbe drasticamente aumentato l'offerta dell'acciaio facendo drasticamente crollare il prezzo, per la nuova impresa fu molto difficile proseguire la produzione e fallí. l'Antitrust americano fece causa ad ALCOA vincendo.
        \subitem \textit{Aumento costi dei beni rivali}: i.e. la concorrenza tra Italo e Frecciarossa. Inizialmente Italo poteva fermarsi solo nelle stazioni secondarie delle cittá, non aveva biglietterie e i treni non venivano annunciati. Antitrust ha permesso a Italo di fare scalo in ogni stazione dopo aver fatto causa a Frecciarossa. 
        \subitem \textit{Ridurre i ricavi dei concorrenti}: come nel punto precedente
\end{enumerate}

\subsubsection{Nuove tecnologie}
Il potere di mercato dipende dalla tecnologia prevalente in un determinato contesto. i.e. Nokia, Samsung ed Apple, le ultime due imprese, entrate nel mercato con nuove tecnologie hanno distrutto Nokia. Nintendo è stata soppiantata per una questione di grafica

\section{Costi sociali del Monopolio}\label{sec:CostiSocialiDelMonopolio}
Qual è il danno che si crea a causa della presenza di un monopolio?
\begin{figure}[H]
    \centering
    \includegraphics[width=.7\linewidth]{images/chapter1/5.jpg}
    \caption{Rappresentazione del DeadWeight Loss}
    \label{fig:dwl}
\end{figure}
\begin{conditions*}
    Q^m &   quantità prodotta in Monopolio\\
    p^m &   Prezzo in monopolio
\end{conditions*}
In concorrenza perfetta si vende di più ad un prezzo più basso ricordando la \ref{eq:MCMR}.

Guardando la figura \ref{fig:dwl} il surplus in \textit{Concorrenza perfetta} è l'insieme delle 3 figure sotto la curva \(P(Q)\). In \textit{Monopolio} invece è l'area \textit{Consumer Surplus} per quanto rigaurda il consumatore, cioe la distanza tra la curva \(P(Q)\) del prezzo e la domanda.

L'area \textit{Monopoly Profit} è invece il \textit{Profitto del monopolista}, una parte del quale, in concorrenza perfetta, era sursplus del consumatore, è quindi presente del surplus che passa dai consumatori all'impresa.

Il triangolo \textit{DWL} è denominato \textit{DDeadweight loss} cioè \textit{Perdita secca di benessere}. rappresenta quindi le risorse che scompaiono dal sistema economico in presenza di monopolio, rappresenza quindi il \textit{Danno del monopolio}.

Calcoliamo \(DWL\):

La base è la differenza trale quantitá vendute in concorrenza perfetta \(Q^S\) e le quantità vendute in monopolio (\(Q^M\)). L'altezza è pari alla differenza dei prezzi tra il prezzo di monopolio (\(P^M\)) e il prezzo in concorrenza perfetta (\(MC\)).

\begin{equation}
    DWL = \frac{1}{2}*\Delta P*\Delta Q
\end{equation}
ricordando le equazioni \ref{eq:elasticita} e \ref{eq:lerner}

Può essere riscritta come segue:
\[
    DWL = \frac{1}{2} * \partial P *\partial Q * (\frac{\partial P}{\partial P})(\frac{P}{P})(\frac{Q}{Q})(\frac{P}{P})
\]
Risolvendo e sostituendo abbiamo il seguente risultato finale (\textit{Herberger's Loss}):
\begin{equation}\label{eq:Herberger}
    DWL = \frac{1}{2}*E_D*L^2*P^M*Q^M
\end{equation}

Guardando questa equazione la perdita dovuta al monopolio sembrerebbe direttamente proporzionale all'Elasticitá della domanda \(E_D\), in realtá, guardando l'equazione rappresentativa di \(L\) (equazione \ref{eq:lerner}) vediamo che l'indice di Lerner \(L\) è il reciproco dell'elasticitá della domanda, l'equazione \ref{eq:Herberger} contiene quindi un errore perchè non esplicita la variabile \(L\). Sapendo inoltre che il profitto è pari al Prezzo al netto dei costi moltiplicato alla quantitá ovvero \(\pi = (P^M - MC)*Q^M\) possiamo riscrivere la \ref{eq:Herberger} come segue:

\begin{equation}
    DWL = \frac{\pi}{2}
\end{equation}

Per calcolare la perdita è sufficiente conoscere il profitto che diventa una misura indiretta sia del potere di mercato sia della perdita sociale dovuta al monopolio. l aperdita di benessere è quindi proporzionale all'extraprofitto\footnote{L'extraprofitto è il profitto in più rispetto a quello normale derivato dalle condizioni finanziarie correnti} dell'impresa.

Collegando il tutto all'economia in presenza di concorrenza perfetta e spaendo che quest'ultima nel lungo periodo presenta un extraprofitto nullo \(\pi = 0\) si dimostra che il potere di mercato da parte delle imprese e conseguente perdita sociale sono nulli ovvero:
\begin{equation}
    \frac{\pi}{2}=\frac{0}{2}=0
\end{equation}

\subsection{Beni durevoli}
Sono beni che vengono comprati una volta e utilizzati più volte. In presenza di beni durevoli l'analisi si complica:
\begin{enumerate}
    \item Il monopolio è \textit{Concorrente di se stesso}: coloro ai quali vende beni durevoli escono dal mercato, la domanda diminuiscre con il tempo e con lei anche il prezzo.
    \item \textit{Deflazione}: Il monopolista tende ad abbassare il prezzo in funzione del tempo, il consumatore ha l'idea che il prezo abbia un andamento decrescente quindi aspetta e non acquista.
\end{enumerate}

\begin{figure}[H]
    \centering
    \includegraphics[width=.7\linewidth]{images/chapter1/6.jpg}
    \caption{Monopolio in presenza di beni durevoli}
    \label{fig:benidurevoli}
\end{figure}
\begin{itemize}
    \item La curva grande \(P(Q)\) è la domanda di mercato
    \item Nel \textit{Periodo 1}
        \subitem si ha \(MC=MR\), il risultato è \(Q_1\), si proietta in alto fino alla curva della domanda \(P(Q)\) per trovare il prezzo di equilibrio \(P_1\)
        \subitem Nel periodo 1 il monopolista ha come area l'insieme dei due rettangoli \(MR(Q_1)\)
    \item Nel \textit{Periodo 2}
        \subitem la domanda scende (infatti la curva della domanda aumenta la sua pendenza, scende più velocemente verso il basso)
        \subitem Il nuovo equilibrio è dato da \(Q_c\), si vende un pò di più ma ad un prezzo \(P_c\) decisamente più basso.
\end{itemize}
La strategia è quindi quella di abbassare i prezzi nel tempo, prende il nome di \textit{Discriminazione intertemporale del prezzo}. Questa strategia è valida per molti settori tra i quali l'editoria e l'alta tecnologia.

\subsection{Congettura di Coase}

Coase mette l'accento sul ruolo dei consumatori, il monopolista ha incentivo a praticare una discriminazione intertemporale del prezzo:

\begin{theorem}
    Il consumatore strategico può dilazionare la decisione di acquistare, sempre che questo non abbia un costo nell'aspettare. La decisione dipende dai tassi di interesse.
\end{theorem}

Se consumare oggi o consumare domani è indifferente per il consumatore allora la congettura di Coase dice che se il monopolista cambia le condizioni del prezzo molto rapidamente allora il consumatore aspetterá fino a quando i prezzi saranno gli stessi della consorrenza perfetta. Di conseguenza il monopolista, in presenza di queste ipotesi, perde il potere di mercato. La congettura di Coase ha significato solo se il consumatore è strategico mentre le imprese non lo sono, si tratta quindi di un'ipotesi poco ragionevole. La presenza di imprese che reagiscono è molto più reale, mentre è davvero poco plausibile che i consumatori siano più strategici.

Ci sono un elenco di strategie per mitigare la congettura di Coase:
\begin{itemize}
    \item Leasing: le rate pagate mensilmente sono relativamente basse, più basse del valore del bene, alla fine del periodo di leasing è possibile riscattare ad un prezzo più alto del valore del bene. In alternativa è possibile sostituire il bene obsoleto con un bene nuovo continuando a pagare le rate anziché riscattare il bene
    \item Investimento in reputazione: ci sono diverse strategie che qualche azienda mette in atto per contrastare la congettura di Coase che intaccano positivamente la reputazione aziendal
        \begin{itemize}
            \item Non si aumenta l'offerta
            \item Non si effettuano sconti
            \item Il prezzo non è discriminato temporalmente
        \end{itemize}
    \item Obsolescenza programmata: tecnica utilizzata soprattutto nell'elettronica secondo cui la durata dei componenti ha una vita limitata. I beni durevoli hanno quindi una durata limitata. La tecnica non è illegale di per se ma solamente se viene nascosto al consumatore e se il consumatore viene ingannato
\end{itemize}

\subsection{Economia di Pacman}

Alcuni studiosi hanno dimostrato che in realtá il comportamento strategico dei consumatori aumenta il potere delle imprese, questo perchè le imprese possono vendere a più persone adattandosi alle esigenze di tutti. Il risultato è l'opposto della congettura di Coase ma le condizioni sono diverse:
\begin{itemize}
    \item le imprese sono strategiche
    \item I consumatori hanno disponibilitá economiche diverse
\end{itemize}

L'ipotesi di pacman è più realistica perché presupporre che tutti i consumatori siano identici non è reale.

\subsubsection{Caso Apple}

La ricerca \textit{Iphone Slow} su Google negli ultimi anni è rappresentata dal seguente grafico:

\begin{figure}[H]
    \centering
    \includegraphics[width=.7\linewidth]{images/chapter1/7.jpg}
    \caption{Ricerca Google \textit{Iphone Slow}}
    \label{fig:Iphoneslow}
\end{figure}

I picchi osservati sono il numero di ricerche \textit{Iphone Slow}, aumentate vertiginosamente poco prima dell'uscita del modello successivo. 

Il caso in questione è stato oggetto di denuncia da parte dell'Antitrust, la strategia commerciale è quella dell'obsolescenza programmata. Oggi l'unione europea cerca di lottare contro il consumismo sfrenato e di obbligare le aziende a produrre beni \textit{riparabili}

\section{Monopolio Naturale}

Nasce per motivi tecnologici ed è spesso associato con la gestione da parte dello stato. Si è visto come il monopolio crei perdita sociale nei capitoli precedenti, soprattutto nella sezione \ref{sec:CostiSocialiDelMonopolio}.

\subsection{Congettura di Hick}

\begin{theorem}
    La cosa migliore di un monopolista è la vita facile
\end{theorem}

Ricerca di una rendita tramite una lobby\footnote{Una lobby è una situazione in cui le imprese riescono a convincere il governo ad avere dei vantaggi dietro pagamentodi ingenti somme di denaro o altri favori}, per mantenere queste lobby occorrono costi elevatissimi

\subsection{Efficienza dinamica: Arrow vs Shumpeter}
\begin{itemize}
    \item Visione di Shumpeter: Il monopolista ha incentivo a investire per guadagnare di più
    \item Visione di Arrow: Il monopolista non ha incentivo a investire ma ha incentivo a sfruttare il più possibile ció che ha, attualmente questa visione è quella più appoggiata
\end{itemize}

Se il monopolista non riesce a risolvere i costi o i prezzi l'unico modo è l'intervento diretto del mercato, tramite la politica di regolazione \textit{ex ante} o tramite la politica dell'Antitrust \textit{ex post}

\subsection{Definizione di monopolio naturale}

\begin{theorem}
    Un'impresa che produce l'intero output del settore ad un costo minore di quello che potrebbe essere il costo se le aziende fossero frammentate. Situazione che si incontra nelle economie di scala, l'impresa spezzata in più parti avrebbe un costo medio più alto che la singola impresa nella sua interezza.
\end{theorem}

\begin{equation}
    \frac{\partial AC(q)}{\partial q} = \frac{C'(q)q-C(q)}{q^2} = \frac{1}{q}\left[C'(q)-\frac{C(q)}{q}\right]
\end{equation}

\begin{conditions*}
    C(q)    &   costo generale dell'impresa\\
    AC(q)   &   \(\frac{C(q)}{q}\)
\end{conditions*}

Quindi \(AC\) è sempre sopra il costo marginale.

\begin{figure}[H]
    \centering
    \includegraphics[width=.7\linewidth]{images/chapter1/8.jpg}
    \caption{Costo medio e costo marginale}
    \label{fig:costomediocostomarginale}
\end{figure}

Il costo è sempre sotto il costo medio, quindi la derivata è negativa e di conseguenza il costo medio è sempre decrescente.

Se l'impresa produce più di un output allora la definizione cambia: per avere un monopolio naturale in uncontesto multiprodotto allora vale la definizione:

\begin{equation}
    C(\sum_{i = 1}^{n} q_i) < \sum_{i = 1}^{n} C(q_i)
\end{equation}

La funzione dei costi è subadditiva cioè se produrre tutti i prodotti insieme costa meno che produrre separatamente, si dimostra che:
\begin{enumerate}
    \item Per avere la condizione di subadditivitá deve esserci la presenza di \textit{economie di scopo}:
        \[C(q_1,0)+C(0,q_2) > C(q_1,q_2)\]
    \item I costi incrementali devono essere decrescenti:
        \[ IC_1(q_1,q_2)=C(q_1,q_2)-C(0,q_2) \]

        \begin{conditions*}
            C(0,q_2)    &   il costo standAlone cioè il costo di realizzare solo il prodotto \(q_2\) e non il prodotto \(q_1\)\\
            AIC &   costi incrementali medi: \(\frac{IC_1(q_1,q_2)}{q_2}\)\\
        \end{conditions*}

        I costi standalone sono più alti di quelli incrementali, i costi standalone sono infatti \textit{upper bound}\footnote{maggiorante cioè un insieme è un qualsiasi elemento che è maggiore o uguale a tutti gli elementi dell'insieme}, mentre quelli incrementali sono \textit{lower bound}\footnote{minorante}, gli incrementali sfruttano l'infrastruttura che già esiste per altri scopi, la struttura standalone invece ha una infrastruttura personale e non condivisa
\end{enumerate}

Ci si chiede quindi dove e in quali situazioni conviene privatizzare le aziende, quando invece creare monopoli o effettuare merges aziendali, quali parti delle industrie regolare e quali laasciare al libero mercato. È possibile pensare che il monopolio natura possa valere anche solo in alcuni segmenti di un industria mentre altri segmenti possono essere lasciati al libero mercato e alla concorrenza, contemporaneamente.

\subsubsection{Esempio: Mercato dell'energia elettrica}

Fasi del mercato dell'energia elettrica:

\begin{figure}[H]
    \centering
    \includegraphics[width=.7\linewidth]{images/chapter1/9.jpg}
    \caption{Fasi del mercato dell'energia elettrica}
\end{figure}


Spinti dall'unione europea volta a liberalizzare i mercati si è deciso di privatizzare determinati settori come quello della produzione dell'energia elettrica rendendo il mercato concorrenziale anche verso i privati stessi che potrebbero autoprocurarsi l'energia elettrica installando, ad esempio, dei pannelli solari. Oggi in Italia TERNA si occupa della distribuzione nazionale dato che la rete è unica. TERNA era una divisione di ENEL, è stata separata e inserita come monopolio, una volta in cittá l'energia è convertita in bassa tensione, anche in queto caso l'energia viene convertita da una sola azienda, si tratta di un monopolio locale e nel caso di Torino è IREN. I venditori finali che forniscono l'energia alle singole case tornano ad essere concorrenti.

Anche il mercato delle telecomunicazioni, soprattutto AT\&T, uno dei giganti delle telecomunicazioni mondiali. Fino al 1980 era il monopolio assoluto negli USA. L'infrastruttura di rete è una enorme barriera all'entrata. L'antitrust americana è intervenuta con uno dei più grandi casi antitrust della storia. La denuncia si basava sul fatto che nel mercato delle telecomunicazioni c'erano grandissimi segmenti che erano ovvi monopoli naturali per la presenza di una sola infrastruttura ma in altri casi la concorrenza era possibile. Venne presa una delle più grandi decisioni della storia per proteggere questo tipo di concorrenza dividendo AT\&T in 8 compagnie diverse, 7 compagnie si occupavano delle comunicazioni urbane in ogni area degli stati uniti di competenza, una sola, sotto il nome di AT\&T, si occupava delle telecomunicazioni interurbane. Con il passare del tempo le aziende che si occupavano di telecomunicazione urbana erano 120 e alla fine degli anni 90, sotto la presidenza Clinton, venne nuovamente deregolarizzato il mercato. Attualmente, con la possibilitá di merge aziendali nel settore le imprese che si occupano di telecumunicazioni negli USA sono rimaste 3: AT\&T, Verizon, Quest.

Da questi casi si ricava che non conta il numero di imprese nel contesto della concorrenza ma conta la quota di mercato che queste imprese detengono. In mercati dove le infrastrutture sono enormi e poche il mercato tende a monopolizzarsi. Dipende quindi dal tipo di infrastruttura.

\subsection{Price Control}

Si cerca ora di dimostrare il motivo per il quale non si fissa un prezzo calmierato in presenza di prezzi considerati troppo elevati

\subsubsection{First Best Pricing}

Questa tecnica consiste nell'imporre che il prezzo eguagli i costi marginali, il problema è che ci si trova in un monopolio naturale quindi i costi medi sono decrescenti e i costi marginali sono sotto i costi medi, si realizza quindi un profitto negativo. Lo stato deve interventire con dei trasferimenti per colmare questo risultato negativo. Accade con determinati servizi come le universitá o i trasporti.

\begin{figure}[H]
    \centering
    \includegraphics[width=.7\linewidth]{images/chapter1/10.jpg}
    \caption{First Best Pricing}
    \label{fig:firstBestPricing}
\end{figure}

La funzione di utilitá quasi lineare del consumatore è:
\begin{align*}
    U^h &= R^h+S^h(p)\\
    \frac{\partial U^h}{\partial p} &= \frac{\partial S^h(p)}{\partial p}
\end{align*}

Per un'impresa monoprodotto:

\begin{align*}
    \max_{\{p\}}W &= S(p)-T+\pi\\
    \pi &= pq(p)+T-C(q(p))-F
\end{align*}
\begin{conditions*}
    W  & Benessere in funzione del prezzo\\
    S(p)-T & surplus netto dei consumatori\\
    \pi & profitto delle imprese in funzione dei prezzi\\
    T & Trasferimenti, vengono estratti dai consumatori (con segno meno nell'equazione di \(W\)) e vengono aggiunti al profitto delle imprese come parte del ricavo (hanno infatti segno positivo in \(\pi\))\\
    F & Costi fissi\\
    C(q(p)) & Costi che dipendono dalla quantità che dipende dal prezzo
\end{conditions*}

Unendo le due funzioni si ottiene la nuova funzione del benessere:

\begin{equation} \label{eq:nuovoBenessere}
    W = S(p)-\bcancel{T} + pq(p) + \bcancel{T} - C(q(p))-F
\end{equation}

si deriva per massimizzare la nuova funzione del benessere \ref{eq:nuovoBenessere}:

\begin{equation}
    \begin{split}
        \frac{\partial W}{\partial p} &= 0 \\
        -\bcancel{q(p)} +\bcancel{q(p)} +pq'(p)-C'(\cdot)q'(p) &= 0\\
        +p\bcancel{q'(p)} &= C'(\cdot)\bcancel{q'(p)}\\
        p &= C'(\cdot)
    \end{split}
\end{equation}

Ancora una volta si nota che il prezzo sociale deve eguagliare il costo marginale per massimizzare il benessere collettivo

\subsubsection{Second Best Pricing}

Si impone il prezzo al costo medio, questa regola prende il nome di \textit{Verage Cost Pricing Rule}, in questo modo il profitto dell'impresa passa da essere negativo ad essere nullo.

\begin{figure}[H]
    \centering
    \includegraphics[width=.7\linewidth]{images/chapter1/11.jpg}
    \caption{Second Best Pricing}
    \label{fig:secondBestPricing}
\end{figure}

Il triangolo tratteggiato in figura \ref{fig:secondBestPricing} rappresenta una perdita di benessere. Tutte le volte che lo stato non pò intervenire allora è costretto ad alzare il prezzo causando una perdita di W.

per un'impresa multiprodotto (supponiamo con prodotto \(1\) e prodotto \(2\)):

\begin{align*}
    C &= F + \sum_{i} c_iq_i\\
    C &= F + c_1q_1 + c_2q_2  
\end{align*}

Il costo dipende dalla distribuzione di F tra il prodotto \(1\) e il prodotto \(2\): solitamente viene utilizzata la tecnica \textit{Full Distributed Cost (FDC)}\footnote{Costi pienamente distribuiti}: Il principio si colloca in un contesto multiprodotto, si vuole che i ricavi del servizio \textit{i-esimo} coprano i costi variabili di quel servizio (oppure i costi diretti\footnote{direttamente imputabili al servizio in questione, per esempio derivanti dal suo diretto utilizzo o sostenuti per poter produrre solo quel bene, non costi collaterali o indirettamente imputabili a quel bene}) e in aggiunta una quota \(f_i\) dei costi fissi complessivi \(F\):

\begin{align*}
    p_iq_i &= c_iq_i + f_iF\\
    p_i &= c_i + \frac{f_iF}{q_i}
\end{align*}
\begin{conditions*}
    f_i & Prende il nome di \textit{Cost Driver} e diventa il fattore di allocazione
\end{conditions*}

Esistono diverse metodologie per calcolare questo \textit{Cost Driver} \(f_i\):

\begin{itemize}
    \item \(\frac{R_i}{\sum_{i = 1}^{n} R_i } \rightarrow\) Metodi di allocaione tramite i ricavi, si attribuisce una quota di costo a ciascun servizio in proporzione alla quota di ricavi che produce il servizio stesso sul totale
    \item \(\frac{Q_i}{\sum_{i = 1}^{n} Q_i } \rightarrow\) Criterio in base alle Quantitá prodotte di quel bene o servizio.
    \item \(\frac{CD_i}{\sum_{i = 1}^{n} CD_i } \rightarrow\) Criterio in base Costi Diretti
\end{itemize}

Si prenda in esame il caso della ripartizione in base ai Volumi \(Q\):
\[
    p_i = c_i + f_i\frac{F}{q_i}    
\]
dove:
\[
    f_i = \frac{q_i}{Q}    
\]
\[
    Q = \sum_{i = 1}^{n} q_i  
\]

di conseguenza:

\begin{align*}
    p_i &= c_i + \frac{q_iF}{Qq_i} \\
    p_i &= c_i + \frac{F}{Q} \\
    p_i - c_i &= \frac{F}{Q} \forall i
\end{align*}
\begin{conditions*}
\forall i & Significa che vale per ogni prodotto
\end{conditions*}

Si dimostra come questa relazione valga per ognuno dei 3 metodi mostrati sopra:

\begin{itemize}
    \item \(\frac{p_i-c_i}{p_i} = \frac{p_j-c_j}{p_j}\)
    \item \(p_i-c_i = p_j-c_j\)
    \item \(\frac{p_i-c_i}{c_i} = \frac{p_j-c_j}{c_j}\)
\end{itemize}


Fin'ora le osservazioni si sono basate solamente sulla domanda, osservando la domanda la soluzione non è quindi efficiente

\begin{figure}[H]
    \centering
    \includegraphics[width=.7\linewidth]{images/chapter1/12.jpg}
    \caption{Perdita secca e extraprofitto al variare di \(E_d\)}
\end{figure}

\begin{conditions*}
    A+B & Extra profitto per coprire i costi fissi \\
    C + D & Perdita secca o \(DWL\)
\end{conditions*}

È possibile ottimizzare il \(DWL\) nel seguente modo
\begin{figure}[H]
    \centering
    \includegraphics[width=.7\linewidth]{images/chapter1/13.jpg}
    \caption{Ottimizzazione DWL}
\end{figure}

\begin{conditions*}
    A'+B' = A+B & Gli extraprofitti sono in media uguali alla soluzione precedente \\
    C'+D' < C+D & La somma delle perdite secce è diminuita, a paritá di extraprofitti medi
\end{conditions*}

In questo modo, giocando sull'elasticitá della domanda, con un prezzo minore in caso di domanda elastica e un prezzo maggiore in caso di domanda rigida si possono mitigare le perdite secche. Si conferma quindi, che anche in caso di monopolio naturale, quindi di monopolio controllato dallo stato, è necessario osservare \(E_d\), nello stesso modo in cui un monopolista non regolato osserva l'indice di Lerner visto nella relazione \ref{eq:indice di lerner}

Si dimostra come trovare il prezzo ottimale:
\begin{align*}
    &\max_{\{p_i\}}S(p_i) + \pi (p_i)\\
    &\text{s.t.} \pi (p_i)>0
\end{align*}
Il profitto non negativo è una condizione necessaria per non mandare l'impresa in perdita, segue che:

\begin{align*}
    L &= S(p_i) + (1+\lambda)\pi (p_i) =\\
        &= S(p_i) + (1+\lambda)\left[ \sum \left( p_iq_i(p_i) - c(q_i(p_i)) \right) \right]
\end{align*}

\begin{conditions*}
    L & Lagrangiana \\
    S(p_i) & Surplus Lordo \\
    \lambda & Prezzo ombra dei fondi pubblici, rappresenta quanto costa alla collettivitá l'uso dei fondi pubblici, dipende dal sistema fiscale in uso. Esiste un indice che mostra quanto è distorsivo il sistema fiscale di un paese, secondo questo indice il paese meno distorto al mondo sono gli USA, negli Stati Uniti esiste il prescreening fiscale per pagare le tasse, inoltre il reato fiscale negli Stati Uniti è un reato penale, la conseguenza è che tutti le pagano\\
\end{conditions*}

A questo punto si deriva:
\begin{align*}
    \frac{\partial L}{\partial p_i} &= \frac{\partial S(p_i)}{\partial p_i} + (1+\lambda)\frac{\partial \pi}{\partial p_i} = \\
    &= -q_i(p_i) + (1+\lambda) \left[ q_i(p_i) + p_iq'_i(p_i) - c'(\cdot)q'_i(p_i) \right] =\\
    &= (1+\lambda)q'_i(p_i)\left[ p_i-c'(\cdot) \right] + \lambda q_i(p_i) =\\
    &= p_i-c'(\cdot) =\\
    &= \frac{-\lambda q_i(p_i)}{(1+\lambda)q'_i(p_i)}
\end{align*}

dividendo per \(p_i\) si trova la condizione di \textit{Ramsey-Boiteaux};

\begin{equation}
    \begin{split}
        L_i &=\frac{p_i-c'(\cdot)}{p_i}=\\
            &=\frac{\lambda}{1+\lambda}\frac{1}{E_{d_i}}
    \end{split}
\end{equation}
Il margine costo-prezzo è inversamente proporzionale a \(E_d\), ricavata nell'equazione \ref{eq:elasticita}, la struttura è la stessa ma cambia il livello tariffario rispetto a Lerner \ref{eq:indice di lerner}, quando il prezzo è fissato dallo stato il prezzo è scalato verso il basso del fattore \(\frac{\lambda}{1+\lambda}\) rispetto a Lerner, il fattore in questione è un fattore \(<1\)

\vspace{2em}
\begin{minipage}[t]{.45\linewidth}
    \textit{Lerner}:
    \[\frac{p_i-c_i'}{p_i}=\frac{1}{E_d}\]
\end{minipage}
\hfill
\vline
\hfill
\begin{minipage}[t]{.45\linewidth}
    \textit{Ramsey-Boiteaux}:
    \[\frac{p_i-c_i'}{p_i}=\frac{1}{E_d}k\]
    dove \(k<1\)
\end{minipage}
\vspace{2em}

La soluzione è diversa per via del fattore \(k\) ma simile. \(k\) prende il nome di \textit{scaling down factor}, lo stato quindi prende il prezzo di monopolio e lo abbassa

\section{Sussidiarietá incrociata}

\subsection{Test di Faulhaber}
per rilevare la sussidiarietá incrociata si usa il test dei prezzi di Faulhaber:

\begin{itemize}
    \item I test dei costi incrementali:
        \begin{itemize}
            \item \(p_1q_1 \geq IC_1(q_1,q_2) = C(q_1,1_2)-C(0,q_2)\)
            \item \(p_2q_2 \geq IC_2(q_1,q_2) = C(q_1,1_2)-C(q_1,0)\)
        \end{itemize}
    \item I test dei costi incrementali:
        \begin{itemize}
            \item \(p_1q_1 \leq C(q_1,0)\)
            \item \(p_2q_2 \leq C(0,q_2)\)
        \end{itemize}
\end{itemize}

Il prezzo deve essere quindi compreso tra il costo incrementale e il costo standalone. Se i due vincoli non sono soddisfatti allora c'è sussidiarietá incrociata

\subsection{Esercizi}

\subsubsection{Esercizio 1}
Si consideri un'impresa che produce 3 prodotti (a,b,c) in modo che sia caratterizzata dalla seguente struttura di costo:
\begin{itemize}
    \item c(a)-c(b)=8
    \item c(c)=6
    \item c(ab)=14
    \item c(bc)=10
    \item c(ac)=10
    \item c(abc)=17
\end{itemize}
Ipotizzando che la struttura dei prezzi che l'impresa intende applicare sia:
\begin{itemize}
    \item p(a)=8
    \item p(b)=6
    \item p(c)=4
\end{itemize}
mostrare se una tale combinazione dei prezzi incorpora effetti di sussidiarizzazione o meno, commentare il risultato.

Con questa tipologia di esercizi si parte dai test di Faulhaber:

I costi incrementali sono:
        \begin{itemize}
            \item I(a)=c(abc)-c(bc)= 7 < p(a)
            \item I(b)=c(abc)-c(ac)= 7 > p(b)
            \item I(c)=c(abc)-c(ab)= 3 < p(c)
        \end{itemize}
Si noti come il prezzo del bene \(b\) sia inferiore al costo incrementale, così come il prezzo del bene \(a\) sia pari al costo standalone. Sembrerebbe dunque che il bene \(a\) sussidi il bene \(b\)

\subsubsection{Esercizio Test 1}
Si immagini un Paese nel quale non vi sono restrizioni di bilancio pubblico e lo Stato può intervenire nel sistema economico usando fondi pubblici. In presenza
di un contesto multiprodotto, un'impresa monopolista regolata vedrà i propri prezzi fissati:
\begin{enumerate}[label=(\alph*)]
    \item in base alla formula di Ramsey
    \item in base alla regola del second best
    \item in base alla regola del first best
    \item in base alla regola dei prezzi pienamente distribuiti
\end{enumerate}
Non essendoci vincoli di bilancio, il regolatore può implementare la soluzione di first best anche in un contesto multiprodotto usando i trasferimenti dal bilancio a compensazione, da cui la risposta corretta è la \(c\). Si ricorda che la regola di Ramsey si applica laddove vi fossero appunto vincoli di bilancio pubblico e quindi l'impossibilità di usare i trasferimenti.

\subsubsection{Esercizio Test 2}
I prezzi regolati alla Ramsey di un monopolio naturale in un contesto multiprodotto prevedono che:
\begin{enumerate}[label=(\alph*)]
    \item il livello dei prezzi sia sempre pari al costo medio di produzione
    \item il livello dei prezzi sia analogo a quello di monopolio non regolato
    \item la struttura dei prezzi sia analoga a quella di un monopolio non regolato
    \item valga la regola di proporzionalità diretta all'elasticità della domanda
\end{enumerate}
I prezzi di Ramsey hanno la stessa struttura dell'indice di Lerner dei prezzi in monopolio moltiplicati per un fattore k < 1. Ciò implica che i prezzi di Ramsey
sono inversamente proporzionali all'elasticità della domanda (ergo la \(d\) è errata), sono superiori ai costi marginali e medi di produzione (ergo la \(a\) è errata, perché vale solo in contesto monoprodotto), hanno un livello più basso dei prezzi di monopolio (altrimenti che senso avrebbe regolarli; ergo la \(b\) è errata), ma la loro struttura è analoga a quella dei prezzi di monopolio non regolato (la \(c\) è l'unica risposta corretta).

\subsubsection{Esercizio test 3}
Un mercato è caratterizzato dalla seguente funzione inversa di domanda: \(p = 120 - q\), dove q è la quantità prodotta. La tecnologia di produzione del bene scambiato su tale mercato è caratterizzata dalla funzione di costo totale \(CT(q) = 6q + 1800\). Nel mercato opera un'impresa monopolista soggetta a regolazione da
parte di un'Autorità settoriale. La soluzione socialmente ottimale di second best è data dalla seguente combinazione (i valori sono da intendersi arrotondati
all'unità):

\begin{enumerate}[label=(\alph*)]
    \item \(p = 6; q = 114\)
    \item \(p = 63; q = 57\)
    \item \(p = 25; q = 95\)
    \item \(p = 101; q = 19\)
\end{enumerate}
La soluzione socialmente ottimale di second best prevede che il prezzo di vendita del servizio sia pari al costo medio di produzione ossia: \(p = 120 - q = 6 + \frac{1800}{q}\) da cui \(q \approx  95\) e \(p \approx  25\). La soluzione in \(1\) è gravemente sbagliata perché corrisponde alla soluzione di first best (p = costo marginale). La soluzione \(b\) è quella che massimizza il profitto del monopolista, mentre la soluzione \(d\) è quella che si deriva dalla condizione di secondo grado sopra riportata, ma rappresenta una soluzione non ottimale dal punto di vista del benessere collettivo.

\section{Esternalitá}

In questo capitolo verrá illustrato come tutte le esternalitá, comprese quelle positive, generano un fallimento di mercato

\subsection{Esternalitá negative}

Ogni volta che si è in presenza di esternalitá i costi marginali esterni incrementano (MEC), sono costi indiretti legati a fattori come l'inquinamento. La somma dei costi marginali di produzione e dei costi marginali esterni danno i costi sociali

\begin{figure}[H]
    \centering
    \includegraphics[width=.7\linewidth]{images/chapter1/14.jpg}
    \caption{Curva considerando solo i MC}
\end{figure}
\begin{conditions*}
    q_1 & è la quantitá venduta dalla singola impresa\\
    p_1 & è il prezzo\\
    D & Domanda\\
    MC & è il costo marginale\\
    S & Offerta
\end{conditions*}

Si introducano ora i MEC:
\begin{figure}[H]
    \centering
    \includegraphics[width=.7\linewidth]{images/chapter1/15.jpg}
    \caption{Cuva considerando ance i MEC}
\end{figure}

E ora si sommino i due costi marginali interni ed esterni:
\begin{figure}[H]
    \centering
    \includegraphics[width=.7\linewidth]{images/chapter1/16.jpg}
    \caption{Curva considerando MEC + MC}
\end{figure}
La soluzione idea dovrebbe tenere conto dei MEC, la quantitá ottimale sarebbe quello di produrre una quantitá \(q^*\), nel mondo reale però non se ne tiene conto, per questo si arriva al fallimento di mercato producendo più di quello che si dovrebbe (si produce \(Q_1\) anzichè \(Q^*\)). A questo punto si noti l'aggregato dei costi sociali che si trova tra \(MSC_1\), \(D\) e la linea tracciata da \(Q_1\) rappresenta il costo di sovrapproduzione

\begin{figure}[H]
    \centering
    \includegraphics[width=.7\linewidth]{images/chapter1/17.jpg}
    \caption{}
\end{figure}
Possibili soluzioni:
\begin{itemize}
    \item La legge impone di usare le nuove tecnologie quindi le imprese devono aumentare la curva dei \(MC\), alzandosi questa curva si riduce l'area dei costi aggregati
    \item La legge impone delle tasse vendendo il cosiddetto \textit{Diritto ad inquinare}, la finalitá è quella di alzare la curva dei \(MC\) sempre riducendo l'area dei costi aggregati sociali
\end{itemize}
Il problema delle politiche ambientali è che diminuiscono il PIL dell'intero paese perchè di fatto le imprese producono meno, sono quindi dette \textit{anticicliche}, cioè vanno bene solo quando l'economia va bene. In un periodo storico molto negativo è lo stato che deve intervenire senza chiedere direttamente alle imprese, lo può fare dando dei sussidi alle imprese per migliorare le tecnologie e diminuire così la curva dei MEC. Un esempio sono gli incentivi ai costruttori di aerei per rinnovare i motori in modo da usare il comburente biofuel, lo stato lo fa per esempio dando sconti agli aerei che ne fanno utilizzo nelle tasse di atterraggio aeroportuali.

\subsection{Esternalitá positive}

Ci sono diveirse esternalitá positive, si terrá conto principalmente delle esternalitá di rete: presenti quando il valore di un bene dipende dal numero di persone che lo usano (rete telefonica, macchine fax, email, internet).
\begin{equation}
    U_i(q)=U_i(N,q)
\end{equation}
dove N è il numero di persone nella rete mentre q è la quantità consumata. Se nessuno utilizza questi beni allora non hanno valore. Le esternalitá di rete sono un tipo di esternalitá \(diretta\) ma esistono anche le esternalità positive di tipo \(indiretto\) ad esempio quando l'utilitá di un gruppo dipende dalla numerostá di un gruppo diverso. Nel caso di esternalitá di tipo diretto il gruppo che consuma il bene dipende dalla numerositá del gruppo stesso
\begin{equation}
    U_{i,A}(q)=U_{i,A}(N_A,q)
\end{equation}
Dove \(A\) è il gruppo in questione. In questi settori gioca un ruolo fondamentale il \textit{positive feedback effect} cioé la numerositá istantanea degli utilizzatori di un determinato bene.

\subsubsection{Metcalfe's law}

\begin{itemize}
    \item Si assuma una rete di \(n\) utenti
    \item Il massimo numero di connessioni è \(n(n-1) = n^2-n\)
    \item quando \(n\leftarrow\infty\) il numero di connessioni può essere approssimato a \(n^2\)
\end{itemize}

\begin{definition}
    La legge di Metcalfe dice che il valore effettivo di una rete è proporzionale al quadrato del numero di connesioni
\end{definition}

\subsubsection{Effetto palla di neve o Positive Feedback Effect}

si supponga che nella rete entri una nuova persona, il nuovo numero di connessioni diventa \((n+1)n=n^2+n\), di conseguenza \(N_{c_0} - N_{c_1} = 2n\), \(2n\) rappresenta l'effetto dell'entrata di un utente nella rete, è quindi un incremento più che proporzionale. Le esternalitá di rete incrementano l'utilitá e incrementando l'utilitá portano ad un'esposizione della domanda.
\subsubsection{Modello di Rohlfs}

Si tratta di un modello utile a capire se la concorrenza perfetta riesce a risolvere gli effetti delle esternalitá. Si tratta di un modello semplificato e applicato principalmente alle reti. \(\Theta\) varia tra 0 e 1 ed è la disponibilitá a pagare, se tende a 0 allora la disponibilitá a pagare è molto alta, se tende a 1 allora la disponibilitá a pagare tende ad essere nulla.

\begin{equation}
    U(\Theta) = \begin{cases}
        n(1-\Theta)-p \text{Se connesso}\\
        0 \text{Se non è connesso}
    \end{cases}
\end{equation}

l'utilitá cresce se il numero di \(n\) tra 0 e 1 cresce, corrisponde al numero delle persone normalizzato a 1.
\begin{align*}
    \text{Beneficio} = n(q-\Theta)\\
    U(\Theta) = n(1-\Theta)-p
\end{align*}

\begin{figure}[H]
    \centering
    \includegraphics[width=.7\linewidth]{images/chapter1/18.jpg}
    \caption{Cruva del beneficio}
\end{figure}
Tutti gli individui che sono tra \(0\) e \(\widetilde{\Theta}\) avranno un beneficio maggiore del prezzo, \(\widetilde{\Theta}\) è quindi il cosiddetto consumatore indifferente, la sua utilitá è pari a 0. Se il prezzo si abbassa aumenteranno gli utilizzatori perchè ci saranno più consumatori con \(Ben>p\). Se si aumenta \(p\) diminuiscono i consumatori:

\begin{figure}[H]
    \centering
    \includegraphics[width=.7\linewidth]{images/chapter1/19.jpg}
    \caption{Aumento del prezzo}
\end{figure}

Si ponga \(n(1-\widetilde{\Theta})-p=0\) e si ricavi \(\widetilde{\Theta}\):

\begin{equation}
    \widetilde{\Theta} = \frac{n-p}{n}
\end{equation}
Rappresenta perciò la domanda di connessioni ovvero il numero potenziale degli utenti che utilizzerebbero il servizio al prezzo \(p\). Il numero effettivo di utenti è \(n\), all'equilibrio quindi \(\widetilde{\Theta} = n\). All'equilibrio si sostituisce quindi \(n\) con \(\widetilde{\Theta}\) dato che sono uguali e si ottiene:
\begin{align*}
    &n(1-\widetilde{\Theta})-p=0\\
    &\widetilde{\Theta}(1-\widetilde{\Theta})-p=0\\
    &\widetilde{\Theta}(1-\widetilde{\Theta})=p
\end{align*}

\begin{figure}[H]
    \centering
    \includegraphics[width=.7\linewidth]{images/chapter1/20.jpg}
    \caption{Benefici degli utilizzatori}
\end{figure}
La paraola rappresenta i benefici degli utilizzatori, \(\widetilde{\Theta^L_0}\) e \(\widetilde{\Theta^H_0}\) sono i due punti di equilibrio, occorre capire quale dei due punti è equilibrio stabile e quale equilibrio instabile. Si analizza ogni punto di equilibrio nel loro intorno:

\begin{figure}[H]
    \centering
    \includegraphics[width=.7\linewidth]{images/chapter1/21.jpg}
    \caption{Analisi di intorno degli equilibri trovati}
\end{figure}

Spostandosi leggeremente a sinistra di \(\widetilde{\Theta^L_0}\) i benefici saranno più bassi quindi tenderá ad andare verso lo 0, se ci si sposta a sinistra allora i benefici continueranno ad aumentare, perció \(\widetilde{\Theta}\) continuerá ad aumentare, l'equilibrio \(\widetilde{\Theta^L_0}\) è instabile. Analizzando invece \(\widetilde{\Theta^H_0}\) si nota che spostandosi a destra il beneficio sará minore del prezzo, spostandosi a sinistra invece tenderá verso il punto \(\widetilde{\Theta^H_0}\), l'equilibrio è quindi considerato stabile.

Da questa semplice dimostrazione se ne trae una regola di fondamentale importanza: al lancio di una nuova tecnologia si deve raggiungere la soglia minima \(\widetilde{\Theta^L_0}\) altrimenti il mercato va a 0. L'elemento più importante non è il tempo ma il prezzo, per questo motivo spesso i servizi sono gratis. Altro elemento fondamentale da tenere in considerazione è il cosidetto \textit{Switching cost}. 

\subsubsection{Analisi di mercato in presenza di esternalitá di rete}

\begin{align*}
    W&=\text{surplus lordo} - P(Q)\times Q + P(Q)\times Q-C(Q) = \\
    &= \text{surplus lordo} - C(Q)
\end{align*}

il surplus lordo vale:
\[\text{surplus lordo} = \int_{0}^{\widehat{\Theta} n(1-\Theta)}  \,d\Theta = n\left(\widehat{\Theta}\frac{\widehat{\Theta}^2}{2}\right) \]
all'equilibrio \(n=\widehat{\Theta}\), la curva sostituendo \(n\) con \(\widehat{\Theta}\) è una curva cubica, aggiungendo all'analisi anche la curva dei costi:

\begin{figure}[H]
    \centering
    \includegraphics[width=.7\linewidth]{images/chapter1/22.jpg}
    \caption{Analisi con curva dei costi}
\end{figure}
Si noti come \(W\) sia massimo quando la distanza tra la curva dei costi e quella del surplus lordo sia massima, si ricordi inoltre che in perfetta concorrenza vale la relazione \(P=C\) di conseguenza \(\widehat{\Theta}(1-\widehat{\Theta})=C\)

\begin{itemize}
    \item \(c=0 \rightarrow \widehat{\Theta_{max}}=1\)
    \item \(c>0 \rightarrow \widehat{\Theta_{max}}<1\)
\end{itemize}
In concorrenza perfetta non si raggiunge l'ottimo in presenza di esternalitá positive di rete a meno che nel caso limite di costo nullo. Quindi sia in presenza di esternalitá negative, sia in presenza di esternalitá positive si raggiunge un fallimento di mercato. Si analizzi ora quale mercato è migliore in presenza di esternalitá di rete positiva, senza raggiungere la soluzione di efficienza.

\subsubsection{Confronto con caso di mercato in Monopolio, con e senza esternalitá di rete}

\begin{figure}[H]
    \centering
    \includegraphics[width=.7\linewidth]{images/chapter1/23.jpg}
    \caption{Analisi con curva dei costi}
\end{figure}
mano a mano che \(c\) sale il monopolista tenderá a coprire sempre meno mercato, la soluzione è quindi sempre e comunque sub-ottimale. La terza colonna rappresnta la penetranza di mercato. Si confronti il caso di monopolio con esternalitá e il monopolio senza esternalitá:

In assenza di esternalitá:
\begin{align*}
                        n   &= 1 \rightarrow \\
    \rightarrow u(\Theta)   &=(1-\Theta)-p \rightarrow \\
    \rightarrow \widehat{\Theta} &= 1-p \rightarrow \\
    \rightarrow \pi_s&=p\times \widehat{\Theta}-c\times \widehat{\Theta} =\\
    &= (1-\widehat{\Theta}\widehat{\Theta}-c\widehat{\Theta}) \rightarrow\\
    \rightarrow \frac{\partial \pi_s}{\partial \widehat{\Theta}} &= 1-2\widehat{\Theta}-c=0
\end{align*}
Senza esternalitá quindi se \(c=0\) il \(\widehat{\Theta_{max}}=\frac{1}{2}\) mentre in presenza di esternalitá \(\widehat{\Theta}=\frac{2}{3}\)

\begin{equation}
    \begin{split}
        & \widehat{\Theta^{\text{no ext pos}}_{max}}=\frac{1}{2}\\
        & \widehat{\Theta^{\text{ext pos}}_{max}}=\frac{2}{3}
    \end{split}
\end{equation}

Anche in presenza di esternalitá di rete il mercato in concorrenza perfetta si comporta comunque in maniera migliore rispetto al mercato in situazione di monopolio.

\subsubsection{Interconnessione della rete tra due imprese vs Acquisto}
Per due reti di dimensione \(n_1\) e \(n_2\) interconnesse come cresce il valore reciproco? Si applica la legge di Metcalfe
\begin{align*}
    \Delta v_1=n_1(n_1+n_2)-n_1^2=n_1n_2\\
    \Delta v_2=n_2(n_1+n_2)-n_2^2=n_1n_2
\end{align*}
Ne consegue che ogni rete ottiene lo stesso valore di interconnessione ma cosa accade se un'impresa acquisisce un'altra impresa?
\[\Delta v_1 = (n_1+n_2)^2 - n_1^2-n_2^2=2n_1n_2\]
L'azienda acquirente ottiene il doppio del beneficio rispetto alla semplice interconnessione
